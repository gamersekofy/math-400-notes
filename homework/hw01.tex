% homework/hw01.tex
\begin{center}
  \section*{Homework 01 - 1.2, 1.3}
  Due Wed 2/5 \\
  Uzair Hamed Mohammed
\end{center}

\subsection*{1.2 Review of Calculus}

\begin{enumerate}
  \item Show that the following equations have at least one solution
    in the given intervals.
    \begin{itemize}
      \item[a.] $x \cos x - 2x^2 + 3x -1 = 0, \hspace{20pt} [0.2,
        0.3] \textrm{ and } [1.2, 1.3]$

        \underline{Sol:} \\
        For interval $[0.2, 0.3]$:
        $$
        \begin{array}{l}
          f(0.2) = 0.2 \cos (0.2) - 2 (0.2)^2 + 3 (0.2) - 1 = -0.284 \\
          f(0.3) = 0.3 \cos (0.3) - 2 (0.3)^2 + 3 (0.3) - 1 = 0.0066
        \end{array}
        $$

        For interval $[1.2, 1.3]$:
        $$
        \begin{array}{l}
          f(1.2) = 1.2 \cos (1.2) - 2 (1.2)^2 + 3 (1.2) - 1 = 0.1548 \\
          f(1.3) = 1.3 \cos (1.3) - 2 (1.3)^2 + 3 (1.3) - 1 = -0.132
        \end{array}
        $$

        Therefore, $x \cos x - 2x^2 + 3x - 1$ has at least one
        solution in both intervals due to sign changes and contunity of f(x)

        \bigbreak

      \item[b.] $(x-2)^2 - \ln x = 0, \hspace{20pt} [1, 2] \textrm{
        and } [e, 4]$

        \underline{Sol:}\\
        For interval $[1, 2]$:
        $$
        \begin{array}{l}
          f(1) = (1-2)^2 - \ln(1) = 1 \\
          f(2) = (2-2)^2 - \ln(2) = -0.693
        \end{array}
        $$

        For interval $[e, 4]$:
        $$
        \begin{array}{l}
          f(e) = (e-2)^2 - \ln(e) = -0.484  \\
          f(4) = (4-2)^2 - \ln(4) = 2.61
        \end{array}
        $$

        Therefore, $(x-2)^2 - \ln x = 0$ has at least one solution in
        both intervals due to sign changes and contunity of f(x)

        \bigbreak

      \item[c.] $2x \cos (2x) - (x-2)^2 = 0, \hspace{20pt} [2, 3]
        \textrm{ and } [3, 4]$

        \underline{Sol:}\\
        For interval $[2, 3]$:
        $$
        \begin{array}{l}
          f(2) = 2(2) \cos(2 \times 2) - (2 - 2)^2 = -2.61\\
          f(3) = 2(3) \cos(2 \times 3) - (3 - 2)^2 = 4.761
        \end{array}
        $$

        For interval $[3, 4]$:
        $$
        \begin{array}{l}
          f(3) = 2(3) \cos(2 \times 3) - (3 - 2)^2 = 4.761  \\
          f(4) = 2(4) \cos(2 \times 4) - (4 - 2)^2 = -5.164
        \end{array}
        $$

        Therefore, $2x \cos (2x) - (x-2)^2 = 0$ has at least one
        solution in both intervals due to sign changes and contunity of f(x)

        \bigbreak
      \item[d.] $x - (\ln x)^x = 0, \hspace{20pt} [4, 5]$

        \underline{Sol:}\\
        For interval $[4, 5]$:
        $$
        \begin{array}{l}
          f(4) = 4 - (\ln 4)^4 = 0.306\\
          f(5) = 5 - (\ln 5)^5 = -5.798
        \end{array}
        $$

        Therefore, $x - (\ln x)^x = 0$ has at least one solution in
        the interval due to sign changes and contunity of f(x)
    \end{itemize}

  \item Find intervals containing solutions to the following equations.
    \begin{itemize}
      \item[a.] $x-3^{-x} = 0$

        \underline{Sol:}\\
        $$
        \begin{array}{l}
          f(0) = 0 - 3^0 = \textrm{-}\\
          f(1) = 1 - 3^{-1} = \textrm{+}
        \end{array}
        $$

        The interval is $[0, 1]$

        \bigbreak
      \item[b.] $4x^2 - e^x = 0$

        \underline{Sol:}\\
        $$
        \begin{array}{l}
          f(0) = 4(0)^2 - e^0 = -\\
          f(1) = 4(1)^2 - e^1 = +
        \end{array}
        $$

        The interval is $[0, 1]$
        \bigbreak

      \item[c.] $x^3 - 2x^2 - 4x + 3 = 0$

        \underline{Sol:}\\
        $$
        \begin{array}{l}
          f(0) = 0^3 - 2*0^2 - 4*0 + 3 = + \\
          f(1) = 1^3 - 2^2 - 4 + 3 = -
        \end{array}
        $$

        The interval is $[0, 1]$

        \bigbreak
      \item[d.] $x^3 = 4.001x^2 + 4.002x = 1.101 = 0$

        \underline{Sol:}\\

        $$
        \begin{array}{l}
          f(-3) = (-3)^3 = 4.001(-3)^2 + 4.002(-3) = 1.101 = -\\
          f(-2) = (-2)^3 = 4.001(-2)^2 + 4.002(-2) = 1.101 = +
        \end{array}
        $$

        The interval is $[-3, -2]$
        \bigbreak

    \end{itemize}

  \item Show that the first derivatives of the following functions
    are zero at least once in the given interests.
    \begin{itemize}
      \item[a.] $f(x) = 1-e^x + (e - 1) \sin(\frac{\pi}{2} x),
        \hspace{20pt} [0, 1]$

        \underline{Sol}:\\
        $$
        \begin{array}{l}
          f(0) = 1-e^0 + (0 - 1) \sin(\frac{\pi}{2} 0) = 0 \\
          f(1) = 1-e^1 + (1 - 1) \sin(\frac{\pi}{2} 1) = 0
        \end{array}
        $$

        Since $f(x)$ is differentiable in the given open interval and
        continuous in the given closed interval, by Rolle's Theorem,
        there exists $c \in (0, 1)$ such that $f'(c) = 0$

        \bigbreak

      \item[b.] $f(x) = (x-1) \tan x + x \sin \pi x, \hspace{20pt} [0, 1]$

        \underline{Sol:}\\
        $$
        \begin{array}{l}
          f(0) = (0-1) \tan 0 + 0 \sin \pi 0 = 0 \\
          f(1) = (1-1) \tan 1 + 1 \sin \pi 1 = 0
        \end{array}
        $$

        Since $f(x)$ is differentiable in the given open interval and
        continuous in the given closed interval, by Rolle's Theorem,
        there exists $c \in (0, 1)$ such that $f'(c) = 0$

        \bigbreak
      \item[c.] $f(x) = x \sin \pi x - (x-2) \ln x, \hspace{20pt} [1, 2]$

        \underline{Sol:}\\
        $$
        \begin{array}{l}
          f(0) = 0 \sin \pi 0 - (0-2) \ln 0 \\
          f(1) = 1 \sin \pi 1 - (1-2) \ln 1
        \end{array}
        $$

        Since $f(x)$ is differentiable in the given open interval and
        continuous in the given closed interval, by Rolle's Theorem,
        there exists $c \in (0, 1)$ such that $f'(c) = 0$

        \bigbreak
      \item[d.] $f(x) = (x - 2) \sin x \ln(x + 2), \hspace{20pt} [-1, 3]$
    \end{itemize}

  \item Find $\textrm{max}_{a \le x \le b} |f(x)|$ for the following
    functions and intervals.
    \begin{itemize}
      \item[a.] $f(x) = \frac{(2-e^x + 2x)}{3}, \hspace{20pt} [0, 1]$

        \underline{Sol:}\\

        $$
        \begin{array}{l}
          f'(x) = \frac{2-e^x}{3}\\
          x = \ln 2\\
          f(0) = \frac{1}{3}\\
          f(1) = \frac{4 - e}{3}
        \end{array}
        $$

        Max = $\frac{2 \ln 2}{3}$

        \bigbreak
      \item[b.] $f(x) = \frac{(4x-3)}{(x^2-2x)}, \hspace{20pt} [0.5, 1]$

        \underline{Sol:}\\

        $$
        \begin{array}{l}
        f'(x) = \frac{-4x^2 + 6x - 6)}{(x^2-2x)^2}\\
        f(0.5) = \frac{4}{3}\\
        f(1) = -1
      \end{array}
      $$

      Max = $\frac{4}{3}$

      \bigbreak
    \item[c.] $f(x) = 2x \cos(2x)-(x-2)^2, \hspace{20pt} [2, 4]$
    \item[d.] $f(x) = 1 + e^{- \cos(x-1)}, \hspace{20pt} [1, 2]$
  \end{itemize}

\item Let $f(x) = x^3$

  \underline{Sol:}\\

  \[
    \begin{array}{l}
      \text{a.} P_2 (x) = 0\\
      \text{b. Error} = 0.125\\
      \text{c.} P_2 (x) = 1 + 3 (x-1) + 3 (x-1)^2\\
      \text{d.} R_2 = -0.125, \text{ actual error} = -0.125
    \end{array}
  \]
\item Let $f(x) = \sqrt{x + 1}$
    
    \underline{Sol:}\\
    \[
      \begin{array}{l}
        \text{a.} P_3 (x) = 1 + \frac{1}{2}x - \frac{1}{8}x^2 + \frac{1}{16} x^3\\
        \text{b.} 0.7109, 0.8662, 1.1182, 1.2344\\
        \text{c.} -0.0038, -0.0002, -0.0002, -0.0097
      \end{array}
    \]
\item Find the second Taylor Polynomial \( P_2(x) \) for the function
  \( f(x) = e^x \cos x \) about \( x_0 = 0 \).
  \begin{itemize}
    \item[a.] Use \( P_2(0.5) \) to approximate \( f(0.5) \). Find an
      upper bound for error \( |f(0.5) - P_2(0.5)| \) using the error
      formula, and compare it to the actual error.

      \underline{Sol:}\\
      \[
        \begin{array}{l}
          P_2(x) = 1 + x \\
          P_2(0.5) = 1.5 \\
          \textrm{Actual } f(0.5) \approx 1.445 \\
          \textrm{Error: } |1.445 - 1.5| = 0.055 \\
          \textrm{Error bound: } \frac{4.473}{6}(0.5)^3 \approx 0.0932
        \end{array}
      \]
      \bigbreak

    \item[b.] Find a bound for the error \( |f(x) - P_2(x)| \) in
      using \( P_2(x) \) to approximate \( f(x) \) on the interval \([0,1]\).

      \underline{Sol:}\\
      \[
        \textrm{Error bound: } \frac{7.525}{6} \cdot 1^3 = 1.254
      \]
      \bigbreak

    \item[c.] Approximate \( \int_0^1 f(x) \, dx \) using \( \int_0^1
      P_2(x) \, dx \).

      \underline{Sol:}\\
      \[
        \int_0^1 P_2(x) \, dx = 1.5 \quad \Rightarrow \quad 1.5
      \]
      \bigbreak

    \item[d.] Find an upper bound for the error in 7c using \(
      \int_0^1 |R_2(x)| \, dx \), and compare the bound to the actual error.

      \underline{Sol:}\\
      \[
        \begin{array}{l}
          \textrm{Error bound: } \frac{7.525}{24} \approx 0.3136 \\
          \textrm{Actual error: } |1.394 - 1.5| = 0.106
        \end{array}
      \]
      \bigbreak
  \end{itemize}

\item Find the Third Taylor polynomial \( P_3(x) \) for the function
  \( f(x) = (x-1) \ln(x) \) about \( x_0 = 1 \).

  \begin{itemize}
    \item[a.] Use \( P_3(0.5) \) to approximate \( f(0.5) \). Find an
      upper bound for error \( |f(0.5) - P_3(0.5)| \) using the error
      formula, and compare it to the actual error.

      \underline{Sol:}\\
      \[
        \begin{array}{l}
          P_3(x) = (x-1)^2 - \frac{1}{2}(x-1)^3 \\
          P_3(0.5) = 0.3125 \\
          \textrm{Actual } f(0.5) \approx 0.3466 \\
          \textrm{Error: } 0.0341 \\
          \textrm{Error bound: } \frac{112}{24} \cdot (0.5)^4 \approx 0.2917
        \end{array}
      \]
      \bigbreak

    \item[b.] Find a bound for the error \( |f(x) - P_3(x)| \) in
      using \( P_3(x) \) to approximate \( f(x) \) on the interval
      \([0.5, 1.5]\).

      \underline{Sol:}\\
      \[
        \textrm{Error bound: } \frac{112}{24} \cdot (0.5)^4 \approx 0.2917
      \]
      \bigbreak

    \item[c.] Approximate \( \int_{0.5}^{1.5} f(x) \, dx \) using \(
      \int_{0.5}^{1.5} P_3(x) \, dx \).

      \underline{Sol:}\\
      \[
        \int_{0.5}^{1.5} P_3(x) \, dx \approx 0.0833
      \]
      \bigbreak

    \item[d.] Find an upper bound for the error in 8c using \(
      \int_{0.5}^{1.5} |R_3(x)| \, dx \), and compare the bound to
      the actual error.

      \underline{Sol:}\\
      \[
        \begin{array}{l}
          \textrm{Error bound: } \approx 0.0583 \\
          \textrm{Actual error: } |0.088 - 0.0833| \approx 0.0047
        \end{array}
      \]
      \bigbreak
  \end{itemize}

\item Use the error term of a Taylor polynomial to estimate the error
  involved in using \( \sin x \approx x \) to approximate \( \sin 1^\circ \).

  \underline{Sol:}\\
  \[
    \begin{array}{l}
      \textrm{Convert } 1^\circ \textrm{ to radians: } x =
      \frac{\pi}{180} \approx 0.0174533. \\
      \textrm{Error term for } P_1(x) = x \textrm{ is } |R_1(x)| \leq
      \frac{|x|^3}{6}. \\
      |R_1| \leq \frac{(\pi/180)^3}{6} \approx 8.85 \times 10^{-7}. \\
      \textrm{Error bound: } \approx 8.85 \times 10^{-7}.
    \end{array}
  \]
  \bigbreak

\item Use a Taylor polynomial about \( \frac{\pi}{4} \) to
  approximate \( \cos 42^\circ \) to an accuracy of \( 10^{-6} \).

  \underline{Sol:}\\
  \[
    \begin{array}{l}
      \textrm{Convert } 42^\circ \textrm{ to radians: } x =
      \frac{7\pi}{30} \approx 0.733. \\
      \textrm{Center at } a = \frac{\pi}{4} \approx 0.785. \\
      \textrm{Compute } |x - a| = \frac{\pi}{60} \approx 0.05236. \\
      \textrm{Find smallest } n \textrm{ such that }
      \frac{(\pi/60)^{n+1}}{(n+1)!} \leq 10^{-6}. \\
      \textrm{For } n = 3: \frac{(0.05236)^4}{24} \approx 3.12 \times
      10^{-7} \leq 10^{-6}. \\
      \textrm{Use } P_3(x) \textrm{ about } \frac{\pi}{4} \textrm{
      with terms up to } (x - \frac{\pi}{4})^3.
    \end{array}
  \]
  \bigbreak

\item Let \( f(x) = e^{x/2} \sin(x/3) \). Determine the following:
  \begin{itemize}
    \item[a.] The third Maclaurin polynomial \( P_3(x) \).

      \underline{Sol:}\\
      \[
        P_3(x) = \frac{x}{3} + \frac{x^2}{6} + \frac{23}{648}x^3
      \]
      \bigbreak

    \item[b.] A bound for the error \( |f(x) - P_3(x)| \) on \([0,1]\).

      \underline{Sol:}\\
      \[
        \textrm{Error bound: } \frac{5}{1296} \approx 0.00386
      \]
      \bigbreak
  \end{itemize}

\item Let \( f(x) = \ln(x^2 + 2) \). Determine the following:
  \begin{itemize}
    \item[a.] The Taylor polynomial \( P_3(x) \) for \( f \) expanded
      about \( x_0 = 1 \).

      \underline{Sol:}\\
      \[
        P_3(x) = \ln 3 + \frac{2}{3}(x-1) + \frac{1}{9}(x-1)^2 +
        \frac{2}{81}(x-1)^3
      \]
      \bigbreak

    \item[b.] The maximum error \( |f(x) - P_3(x)| \) for \( 0 \leq x \leq 1 \).

      \underline{Sol:}\\
      \[
        \textrm{Error bound: } 0.125
      \]
      \bigbreak

    \item[c.] The Maclaurin polynomial \( \tilde{P}_3(x) \) for \( f \).

      \underline{Sol:}\\
      \[
        \tilde{P}_3(x) = \ln 2 + \frac{x^2}{2}
      \]
      \bigbreak

    \item[d.] The maximum error \( |f(x) - \tilde{P}_3(x)| \) for \(
      0 \leq x \leq 1 \).

      \underline{Sol:}\\
      \[
        \textrm{Error bound: } 0.125
      \]
      \bigbreak

    \item[e.] Does \( P_3(0) \) approximate \( f(0) \) better than \(
      \tilde{P}_3(1) \) approximates \( f(1) \)?

      \underline{Sol:}\\
      \[
        \begin{array}{l}
          \textrm{Error at } P_3(0): |\ln 2 - 0.5183| \approx 0.1748 \\
          \textrm{Error at } \tilde{P}_3(1): |\ln 3 - 1.1931| \approx 0.0945 \\
          \textrm{No, } \tilde{P}_3(1) \textrm{ approximates } f(1)
          \textrm{ better.}
        \end{array}
      \]
      \bigbreak
  \end{itemize}

\item Find a bound for the maximum error when using \( P_2(x) = 1 -
  \frac{1}{2}x^2 \) to approximate \( f(x) = \cos x \) on
  \(\left[-\frac{1}{2}, \frac{1}{2}\right]\).

  \underline{Sol:}\\
  \[
    \begin{array}{l}
      \textrm{Error term: } R_2(x) = \frac{f^{(4)}(c)}{4!}x^4 \quad
      (c \in [-1/2, 1/2]) \\
      \textrm{Since } f^{(4)}(x) = \cos x \textrm{, } |f^{(4)}(c)| \leq 1 \\
      \textrm{Max } |x|^4 \leq \left(\frac{1}{2}\right)^4 = \frac{1}{16} \\
      \textrm{Error bound: } |R_2(x)| \leq \frac{1}{24} \cdot
      \frac{1}{16} = \frac{1}{384} \approx 0.0026
    \end{array}
  \]
  \bigbreak
\item The \( n \)-th Taylor polynomial for a function \( f \) at \(
  x_0 \) is sometimes referred to as the polynomial of degree at most
  \( n \) that best approximates \( f \) near \( x_0 \).
  \begin{itemize}
    \item[a.] Explain why this description is accurate.

      \underline{Sol:}\\
      The \( n \)-th Taylor polynomial \( P_n(x) \) matches \( f \)
      and its first \( n \) derivatives at \( x_0 \). This ensures
      the polynomial shares the function’s value, slope, curvature,
      and higher-order behaviors at \( x_0 \), minimizing the
      approximation error near \( x_0 \). The error \( |f(x) -
      P_n(x)| \) grows only with \( |x - x_0|^{n+1} \), making \(
      P_n(x) \) the "best" local approximation among polynomials of
      degree \( \leq n \).
      \bigbreak

    \item[b.] Find the quadratic polynomial that best approximates a
      function \( f \) near \( x_0 = 1 \) if the tangent line at \(
      x_0 = 1 \) has equation \( y = 4x - 1 \), and \( f''(1) = 6 \).

      \underline{Sol:}\\
      \[
        \begin{array}{l}
          \textrm{From the tangent line: } f(1) = 3, \quad f'(1) = 4. \\
          \textrm{Quadratic polynomial: } \\
          P_2(x) = f(1) + f'(1)(x-1) + \frac{f''(1)}{2}(x-1)^2 \\
          P_2(x) = 3 + 4(x-1) + 3(x-1)^2.\\


        \end{array}
      \]
      \bigbreak
  \end{itemize}

\item The error function is defined by
  \[
    \operatorname{erf}(x) = \frac{2}{\sqrt{\pi}} \int_0^x e^{-t^2} \, dt.
  \]
  \begin{itemize}
    \item[a.] Integrate the Maclaurin series for \( e^{-t^2} \) to show that
      \[
        \operatorname{erf}(x) = \frac{2}{\sqrt{\pi}}
        \sum_{k=0}^{\infty} \frac{(-1)^k x^{2k+1}}{(2k+1)k!}.
      \]

      \underline{Sol:}\\
      \[
        \begin{array}{l}
          \textrm{Maclaurin series: } e^{-t^2} = \sum_{k=0}^\infty
          \frac{(-1)^k t^{2k}}{k!}. \\
          \textrm{Integrate term-by-term: } \int_0^x e^{-t^2} dt =
          \sum_{k=0}^\infty \frac{(-1)^k}{(2k+1)k!} x^{2k+1}. \\
          \textrm{Multiply by } \frac{2}{\sqrt{\pi}} \textrm{ to
          obtain the series.}
        \end{array}
      \]
      \bigbreak

    \item[b.] Verify that the two series agree for \( k = 1, 2, 3, 4 \).

      \underline{Sol:}\\
      \[
        \begin{array}{l}
          \textrm{Expand both series up to } k=4: \\
          \textrm{Series (a): } \frac{2}{\sqrt{\pi}} \left( x -
            \frac{x^3}{3} + \frac{x^5}{10} - \frac{x^7}{42} +
          \frac{x^9}{216} \right). \\
          \textrm{Series (b): } \frac{2}{\sqrt{\pi}} e^{-x^2} \left(
            x + \frac{2x^3}{3} + \frac{4x^5}{15} + \frac{8x^7}{105} +
          \frac{16x^9}{945} \right). \\
          \textrm{Multiply } e^{-x^2} \approx 1 - x^2 + \frac{x^4}{2}
          - \frac{x^6}{6} + \frac{x^8}{24} \textrm{ into series (b):} \\
          \textrm{Result matches series (a) up to } x^9 \textrm{
          (coefficients agree).}
        \end{array}
      \]
      \bigbreak

    \item[c.] Approximate \( \operatorname{erf}(1) \) to within \( 10^{-7} \).

      \underline{Sol:}\\
      \[
        \begin{array}{l}
          \textrm{Compute terms until } \frac{2}{\sqrt{\pi}} \cdot
          \frac{1}{(2k+1)k!} < 10^{-7}. \\
          \textrm{At } k=6: \frac{2}{\sqrt{\pi}} \cdot \frac{1}{13
          \cdot 6!} \approx 1.08 \times 10^{-8} < 10^{-7}. \\
          \operatorname{erf}(1) \approx 0.84270079.
        \end{array}
      \]
      \bigbreak

    \item[d.] Use the same number of terms (\( k=6 \)) with the
      series in part (b).

      \underline{Sol:}\\
      \[
        \textrm{Approximation: } \operatorname{erf}(1) \approx
        0.84270079 \quad \textrm{(same accuracy as part c).}
      \]
      \bigbreak

    \item[e.] Explain difficulties using the series in part (b).

      \underline{Sol:}\\
      \[
        \textrm{Series (b) requires multiplying two infinite series,
          leading to computational complexity and potential loss of
          precision due to alternating signs. Additionally, terms grow
        before decaying, causing numerical instability.}
      \]
      \bigbreak
  \end{itemize}

\item Verify that \( |\sin x| \leq |x| \) for all \( x \).
  \begin{itemize}
    \item[a.] Show that for \( x \geq 0 \), \( f(x) = x - \sin x \)
      is non-decreasing, implying \( \sin x \leq x \).

      \underline{Sol:}\\
      \[
        \begin{array}{l}
          f'(x) = 1 - \cos x \geq 0 \quad (\text{since } \cos x \leq
          1 \text{ for all } x). \\
          \Rightarrow f(x) \text{ is non-decreasing on } [0, \infty). \\
          \text{At } x = 0: f(0) = 0 - \sin 0 = 0. \\
          \text{For } x \geq 0: f(x) \geq f(0) \implies x - \sin x
          \geq 0 \implies \sin x \leq x.
        \end{array}
      \]
      \bigbreak

    \item[b.] Conclude using \( \sin(-x) = -\sin x \).

      \underline{Sol:}\\
      \[
        \begin{array}{l}
          \text{For } x < 0: \\
          |\sin x| = |\sin(-x)| = |-\sin(-x)| = |\sin(-x)| \leq |-x|
          = |x| \quad (\text{by part (a)}). \\
          \text{Thus, } |\sin x| \leq |x| \text{ for all } x \in \mathbb{R}.
        \end{array}
      \]
      \bigbreak
  \end{itemize}

\end{enumerate}

\subsection*{1.3 Round-Off Error and Computer Arithmetic}

\begin{enumerate}
\item Compute the absolute error and relative error in approximations
  of \( p \) by \( p^* \).
  \begin{itemize}
    \item[a.] \( p = \pi \), \( p^* = \frac{22}{7} \)

      \underline{Sol:}\\
      \[
        \begin{array}{l}
          \textrm{Absolute error: } \left| \pi - \frac{22}{7} \right|
          \approx 0.001264 \\
          \textrm{Relative error: } \frac{0.001264}{\pi} \approx
          0.000402 \quad (0.0402\%)
        \end{array}
      \]
      \bigbreak

    \item[b.] \( p = \pi \), \( p^* = 3.1416 \)

      \underline{Sol:}\\
      \[
        \begin{array}{l}
          \textrm{Absolute error: } \left| \pi - 3.1416 \right|
          \approx 0.00000735 \\
          \textrm{Relative error: } \frac{0.00000735}{\pi} \approx
          0.00000234 \quad (0.000234\%)
        \end{array}
      \]
      \bigbreak

    \item[c.] \( p = e \), \( p^* = 2.718 \)

      \underline{Sol:}\\
      \[
        \begin{array}{l}
          \textrm{Absolute error: } \left| e - 2.718 \right| \approx
          0.0002818 \\
          \textrm{Relative error: } \frac{0.0002818}{e} \approx
          0.0001037 \quad (0.01037\%)
        \end{array}
      \]
      \bigbreak

    \item[d.] \( p = \sqrt{2} \), \( p^* = 1.414 \)

      \underline{Sol:}\\
      \[
        \begin{array}{l}
          \textrm{Absolute error: } \left| \sqrt{2} - 1.414 \right|
          \approx 0.0002136 \\
          \textrm{Relative error: } \frac{0.0002136}{\sqrt{2}}
          \approx 0.000151 \quad (0.0151\%)
        \end{array}
      \]
      \bigbreak

    \item[e.] \( p = e^{10} \), \( p^* = 22000 \)

      \underline{Sol:}\\
      \[
        \begin{array}{l}
          \textrm{Absolute error: } \left| e^{10} - 22000 \right|
          \approx 26.4658 \\
          \textrm{Relative error: } \frac{26.4658}{e^{10}} \approx
          0.001201 \quad (0.1201\%)
        \end{array}
      \]
      \bigbreak

    \item[f.] \( p = 10^{\pi} \), \( p^* = 1400 \)

      \underline{Sol:}\\
      \[
        \begin{array}{l}
          \textrm{Absolute error: } \left| 10^{\pi} - 1400 \right| \approx 15 \\
          \textrm{Relative error: } \frac{15}{10^{\pi}} \approx
          0.01083 \quad (1.083\%)
        \end{array}
      \]
      \bigbreak

    \item[g.] \( p = 8! \), \( p^* = 39900 \)

      \underline{Sol:}\\
      \[
        \begin{array}{l}
          \textrm{Absolute error: } \left| 40320 - 39900 \right| = 420 \\
          \textrm{Relative error: } \frac{420}{40320} \approx 0.0104
          \quad (1.04\%)
        \end{array}
      \]
      \bigbreak

    \item[h.] \( p = 9! \), \( p^* = \sqrt{18\pi} \left( \frac{9}{e}
      \right)^9 \)

      \underline{Sol:}\\
      \[
        \begin{array}{l}
          \textrm{Absolute error: } \left| 362880 - 359500 \right|
          \approx 3380 \\
          \textrm{Relative error: } \frac{3380}{362880} \approx
          0.00931 \quad (0.931\%)
        \end{array}
      \]
      \bigbreak
  \end{itemize}

\item Perform the following computations (i) exactly, (ii) using
  three-digit chopping arithmetic, and (iii) using three-digit
  rounding arithmetic. (iv) Compute the relative errors in (ii) and (iii).
  \begin{itemize}
    \item[a.] \( \frac{4}{5} + \frac{1}{3} \)

      \underline{Sol:}\\
      \[
        \begin{array}{l}
          \textrm{(i) Exact: } \frac{17}{15} \approx 1.1333333333 \\
          \textrm{(ii) Chopping: } 1.13 \\
          \textrm{(iii) Rounding: } 1.13 \\
          \textrm{(iv) Relative errors: } 0.294\% \quad \textrm{(both)}
        \end{array}
      \]
      \bigbreak

    \item[b.] \( \frac{4}{5} \times \frac{1}{3} \)

      \underline{Sol:}\\
      \[
        \begin{array}{l}
          \textrm{(i) Exact: } \frac{4}{15} \approx 0.2666666667 \\
          \textrm{(ii) Chopping: } 0.266 \\
          \textrm{(iii) Rounding: } 0.266 \\
          \textrm{(iv) Relative errors: } 0.25\% \quad \textrm{(both)}
        \end{array}
      \]
      \bigbreak

    \item[c.] \( \left( \frac{1}{3} - \frac{3}{11} \right) + \frac{3}{20} \)

      \underline{Sol:}\\
      \[
        \begin{array}{l}
          \textrm{(i) Exact: } \frac{139}{660} \approx 0.2106060606 \\
          \textrm{(ii) Chopping: } 0.211 \quad \textrm{Error: } 0.187\% \\
          \textrm{(iii) Rounding: } 0.210 \quad \textrm{Error: } 0.288\% \\
        \end{array}
      \]
      \bigbreak

    \item[d.] \( \left( \frac{1}{3} + \frac{3}{11} \right) - \frac{3}{20} \)

      \underline{Sol:}\\
      \[
        \begin{array}{l}
          \textrm{(i) Exact: } \frac{301}{660} \approx 0.4560606061 \\
          \textrm{(ii) Chopping: } 0.455 \quad \textrm{Error: } 0.232\% \\
          \textrm{(iii) Rounding: } 0.456 \quad \textrm{Error: } 0.0133\% \\
        \end{array}
      \]
      \bigbreak
  \end{itemize}

\item Perform the following computations using three-digit rounding
  arithmetic and compute errors.
  \begin{itemize}
    \item[a.] \( 133 + 0.921 \)

      \underline{Sol:}\\
      \[
        \begin{array}{l}
          \textrm{Exact: } 133.921 \\
          \textrm{Approx: } 134 \\
          \textrm{Absolute error: } 0.079 \\
          \textrm{Relative error: } 0.0590\%
        \end{array}
      \]
      \bigbreak

    \item[b.] \( 133 - 0.499 \)

      \underline{Sol:}\\
      \[
        \begin{array}{l}
          \textrm{Exact: } 132.501 \\
          \textrm{Approx: } 133 \\
          \textrm{Absolute error: } 0.499 \\
          \textrm{Relative error: } 0.376\%
        \end{array}
      \]
      \bigbreak

    \item[c.] \( (121 - 0.327) - 119 \)

      \underline{Sol:}\\
      \[
        \begin{array}{l}
          \textrm{Exact: } 1.673 \\
          \textrm{Approx: } 2.00 \\
          \textrm{Absolute error: } 0.327 \\
          \textrm{Relative error: } 19.5\%
        \end{array}
      \]
      \bigbreak

    \item[d.] \( (121 - 119) - 0.327 \)

      \underline{Sol:}\\
      \[
        \begin{array}{l}
          \textrm{Exact: } 1.673 \\
          \textrm{Approx: } 1.67 \\
          \textrm{Absolute error: } 0.003 \\
          \textrm{Relative error: } 0.179\%
        \end{array}
      \]
      \bigbreak

    \item[e.] \( \frac{\frac{13}{14} - \frac{6}{7}}{2e - 5.4} \)

      \underline{Sol:}\\
      \[
        \begin{array}{l}
          \textrm{Exact: } \approx 1.9528 \\
          \textrm{Approx: } 1.80 \\
          \textrm{Absolute error: } 0.1528 \\
          \textrm{Relative error: } 7.82\%
        \end{array}
      \]
      \bigbreak

    \item[f.] \( -10\pi + 6e - \frac{3}{62} \)

      \underline{Sol:}\\
      \[
        \begin{array}{l}
          \textrm{Exact: } \approx -15.1546 \\
          \textrm{Approx: } -15.1 \\
          \textrm{Absolute error: } 0.0546 \\
          \textrm{Relative error: } 0.360\%
        \end{array}
      \]
      \bigbreak

    \item[g.] \( \left( \frac{2}{9} \right) \times \left( \frac{9}{7} \right) \)

      \underline{Sol:}\\
      \[
        \begin{array}{l}
          \textrm{Exact: } \approx 0.2857 \\
          \textrm{Approx: } 0.286 \\
          \textrm{Absolute error: } 0.000286 \\
          \textrm{Relative error: } 0.0999\%
        \end{array}
      \]
      \bigbreak

    \item[h.] \( \frac{\pi - \frac{22}{7}}{\frac{1}{17}} \)

      \underline{Sol:}\\
      \[
        \begin{array}{l}
          \textrm{Exact: } \approx -0.0215 \\
          \textrm{Approx: } 0.00 \\
          \textrm{Absolute error: } 0.0215 \\
          \textrm{Relative error: } 100\%
        \end{array}
      \]
      \bigbreak
  \end{itemize}

\item Repeat question 3 using three-digit chopping arithmetic.
  \begin{itemize}
    \item[a.] \( 133 + 0.921 \)

      \underline{Sol:}\\
      \[
        \begin{array}{l}
          \textrm{Exact: } 133.921 \\
          \textrm{Chopped: } 133 \\
          \textrm{Absolute error: } 0.921 \\
          \textrm{Relative error: } 0.688\%
        \end{array}
      \]
      \bigbreak

    \item[b.] \( 133 - 0.499 \)

      \underline{Sol:}\\
      \[
        \begin{array}{l}
          \textrm{Exact: } 132.501 \\
          \textrm{Chopped: } 132 \\
          \textrm{Absolute error: } 0.501 \\
          \textrm{Relative error: } 0.378\%
        \end{array}
      \]
      \bigbreak

    \item[c.] \( (121 - 0.327) - 119 \)

      \underline{Sol:}\\
      \[
        \begin{array}{l}
          \textrm{Exact: } 1.673 \\
          \textrm{Chopped: } 1.00 \\
          \textrm{Absolute error: } 0.673 \\
          \textrm{Relative error: } 40.2\%
        \end{array}
      \]
      \bigbreak

    \item[d.] \( (121 - 119) - 0.327 \)

      \underline{Sol:}\\
      \[
        \begin{array}{l}
          \textrm{Exact: } 1.673 \\
          \textrm{Chopped: } 1.67 \\
          \textrm{Absolute error: } 0.003 \\
          \textrm{Relative error: } 0.179\%
        \end{array}
      \]
      \bigbreak

    \item[e.] \( \frac{\frac{13}{14} - \frac{6}{7}}{2e - 5.4} \)

      \underline{Sol:}\\
      \[
        \begin{array}{l}
          \textrm{Exact: } \approx 1.9528 \\
          \textrm{Chopped: } 2.36 \\
          \textrm{Absolute error: } 0.4072 \\
          \textrm{Relative error: } 20.8\%
        \end{array}
      \]
      \bigbreak

    \item[f.] \( -10\pi + 6e - \frac{3}{62} \)

      \underline{Sol:}\\
      \[
        \begin{array}{l}
          \textrm{Exact: } \approx -15.1546 \\
          \textrm{Chopped: } -15.1 \\
          \textrm{Absolute error: } 0.0546 \\
          \textrm{Relative error: } 0.360\%
        \end{array}
      \]
      \bigbreak

    \item[g.] \( \left( \frac{2}{9} \right) \times \left( \frac{9}{7} \right) \)

      \underline{Sol:}\\
      \[
        \begin{array}{l}
          \textrm{Exact: } \approx 0.2857 \\
          \textrm{Chopped: } 0.284 \\
          \textrm{Absolute error: } 0.0017 \\
          \textrm{Relative error: } 0.599\%
        \end{array}
      \]
      \bigbreak

    \item[h.] \( \frac{\pi - \frac{22}{7}}{\frac{1}{17}} \)

      \underline{Sol:}\\
      \[
        \begin{array}{l}
          \textrm{Exact: } \approx -0.0215 \\
          \textrm{Chopped: } -0.017 \\
          \textrm{Absolute error: } 0.0045 \\
          \textrm{Relative error: } 20.9\%
        \end{array}
      \]
      \bigbreak
  \end{itemize}

\item Repeat question 3 using four-digit rounding arithmetic.
  \begin{itemize}
    \item[a.] \( 133 + 0.921 \)

      \underline{Sol:}\\
      \[
        \begin{array}{l}
          \textrm{Exact: } 133.921 \\
          \textrm{Approx: } 133.9 \\
          \textrm{Absolute error: } 0.021 \\
          \textrm{Relative error: } 0.0157\%
        \end{array}
      \]
      \bigbreak

    \item[b.] \( 133 - 0.499 \)

      \underline{Sol:}\\
      \[
        \begin{array}{l}
          \textrm{Exact: } 132.501 \\
          \textrm{Approx: } 132.5 \\
          \textrm{Absolute error: } 0.001 \\
          \textrm{Relative error: } 0.000755\%
        \end{array}
      \]
      \bigbreak

    \item[c.] \( (121 - 0.327) - 119 \)

      \underline{Sol:}\\
      \[
        \begin{array}{l}
          \textrm{Exact: } 1.673 \\
          \textrm{Approx: } 1.700 \\
          \textrm{Absolute error: } 0.027 \\
          \textrm{Relative error: } 1.614\%
        \end{array}
      \]
      \bigbreak

    \item[d.] \( (121 - 119) - 0.327 \)

      \underline{Sol:}\\
      \[
        \begin{array}{l}
          \textrm{Exact: } 1.673 \\
          \textrm{Approx: } 1.673 \\
          \textrm{Absolute error: } 0 \\
          \textrm{Relative error: } 0\%
        \end{array}
      \]
      \bigbreak

    \item[e.] \( \frac{\frac{13}{14} - \frac{6}{7}}{2e - 5.4} \)

      \underline{Sol:}\\
      \[
        \begin{array}{l}
          \textrm{Exact: } \approx 1.9538 \\
          \textrm{Approx: } 1.932 \\
          \textrm{Absolute error: } 0.0218 \\
          \textrm{Relative error: } 1.115\%
        \end{array}
      \]
      \bigbreak

    \item[f.] \( -10\pi + 6e - \frac{3}{62} \)

      \underline{Sol:}\\
      \[
        \begin{array}{l}
          \textrm{Exact: } \approx -15.1546 \\
          \textrm{Approx: } -15.16 \\
          \textrm{Absolute error: } 0.0054 \\
          \textrm{Relative error: } 0.0356\%
        \end{array}
      \]
      \bigbreak

    \item[g.] \( \left( \frac{2}{9} \right) \times \left( \frac{9}{7} \right) \)

      \underline{Sol:}\\
      \[
        \begin{array}{l}
          \textrm{Exact: } \approx 0.285714 \\
          \textrm{Approx: } 0.2857 \\
          \textrm{Absolute error: } 0.000014 \\
          \textrm{Relative error: } 0.0049\%
        \end{array}
      \]
      \bigbreak

    \item[h.] \( \frac{\pi - \frac{22}{7}}{\frac{1}{17}} \)

      \underline{Sol:}\\
      \[
        \begin{array}{l}
          \textrm{Exact: } \approx -0.0215 \\
          \textrm{Approx: } -0.01700 \\
          \textrm{Absolute error: } 0.0045 \\
          \textrm{Relative error: } 20.93\%
        \end{array}
      \]
      \bigbreak
  \end{itemize}

\item Repeat question 3 using four-digit chopping arithmetic.
  \begin{itemize}
    \item[a.] \( 133 + 0.921 \)

      \underline{Sol:}\\
      \[
        \begin{array}{l}
          \textrm{Exact: } 133.921 \\
          \textrm{Chopped: } 133.9 \\
          \textrm{Absolute error: } 0.021 \\
          \textrm{Relative error: } 0.0157\%
        \end{array}
      \]
      \bigbreak

    \item[b.] \( 133 - 0.499 \)

      \underline{Sol:}\\
      \[
        \begin{array}{l}
          \textrm{Exact: } 132.501 \\
          \textrm{Chopped: } 132.5 \\
          \textrm{Absolute error: } 0.001 \\
          \textrm{Relative error: } 0.000755\%
        \end{array}
      \]
      \bigbreak

    \item[c.] \( (121 - 0.327) - 119 \)

      \underline{Sol:}\\
      \[
        \begin{array}{l}
          \textrm{Exact: } 1.673 \\
          \textrm{Chopped: } 1.600 \\
          \textrm{Absolute error: } 0.073 \\
          \textrm{Relative error: } 4.36\%
        \end{array}
      \]
      \bigbreak

    \item[d.] \( (121 - 119) - 0.327 \)

      \underline{Sol:}\\
      \[
        \begin{array}{l}
          \textrm{Exact: } 1.673 \\
          \textrm{Chopped: } 1.673 \\
          \textrm{Absolute error: } 0 \\
          \textrm{Relative error: } 0\%
        \end{array}
      \]
      \bigbreak

    \item[e.] \( \frac{\frac{13}{14} - \frac{6}{7}}{2e - 5.4} \)

      \underline{Sol:}\\
      \[
        \begin{array}{l}
          \textrm{Exact: } \approx 1.9538 \\
          \textrm{Chopped: } 1.983 \\
          \textrm{Absolute error: } 0.0292 \\
          \textrm{Relative error: } 1.5\%
        \end{array}
      \]
      \bigbreak

    \item[f.] \( -10\pi + 6e - \frac{3}{62} \)

      \underline{Sol:}\\
      \[
        \begin{array}{l}
          \textrm{Exact: } \approx -15.1553 \\
          \textrm{Chopped: } -15.15 \\
          \textrm{Absolute error: } 0.0053 \\
          \textrm{Relative error: } 0.035\%
        \end{array}
      \]
      \bigbreak

    \item[g.] \( \left( \frac{2}{9} \right) \times \left( \frac{9}{7} \right) \)

      \underline{Sol:}\\
      \[
        \begin{array}{l}
          \textrm{Exact: } \approx 0.2857 \\
          \textrm{Chopped: } 0.2856 \\
          \textrm{Absolute error: } 0.000114 \\
          \textrm{Relative error: } 0.04\%
        \end{array}
      \]
      \bigbreak

    \item[h.] \( \frac{\pi - \frac{22}{7}}{\frac{1}{17}} \)

      \underline{Sol:}\\
      \[
        \begin{array}{l}
          \textrm{Exact: } \approx -0.0215 \\
          \textrm{Chopped: } -0.017 \\
          \textrm{Absolute error: } 0.0045 \\
          \textrm{Relative error: } 20.9\%
        \end{array}
      \]
      \bigbreak
  \end{itemize}

\item Compute the absolute error and relative error in approximations
  of \( \pi \) using the given formulas with the Maclaurin polynomial
  for \( \arctan x \).
  \begin{itemize}
    \item[a.] \( 4\left[\arctan\left(\frac{1}{2}\right) +
      \arctan\left(\frac{1}{3}\right)\right] \)

      \underline{Sol:}\\
      \[
        \begin{array}{l}
          \textrm{Approximation: } 4\left[\left(\frac{1}{2} -
            \frac{1}{24} + \frac{1}{160}\right) + \left(\frac{1}{3} -
          \frac{1}{81} + \frac{1}{1215}\right)\right] \approx 3.1456 \\
          \textrm{Absolute error: } |\pi - 3.1456| \approx 0.00398 \\
          \textrm{Relative error: } \frac{0.00398}{\pi} \approx 0.1268\%
        \end{array}
      \]
      \bigbreak

    \item[b.] \( 14\arctan\left(\frac{1}{5}\right) -
      4\arctan\left(\frac{1}{239}\right) \)

      \underline{Sol:}\\
      \[
        \begin{array}{l}
          \textrm{Approximation: } 16\left(\frac{1}{5} -
            \frac{1}{3}\left(\frac{1}{5}\right)^3 +
          \frac{1}{5}\left(\frac{1}{5}\right)^5\right) -
          4\left(\frac{1}{239} -
            \frac{1}{3}\left(\frac{1}{239}\right)^3 +
          \frac{1}{5}\left(\frac{1}{239}\right)^5\right) \approx 3.1416 \\
          \textrm{Absolute error: } |\pi - 3.1416| = -2.83757402069e^{-05} \\
          \textrm{Relative error: } \frac{3.1416}{\pi} = -9.03227863564e^{-06}\%
        \end{array}
      \]
      \bigbreak
  \end{itemize}
\end{enumerate}
