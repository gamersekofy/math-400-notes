\begin{center}
  \section*{Homework 12 - 7.4, 7.6}
  Due Wed 4/30 \\
  Uzair Hamed Mohammed
\end{center}

\subsection*{7.4 The Jacobi and Gauss-Seidel Methods}

1a, 2a, 4a, 5

\begin{enumerate}
  \item[1.] Find the first 2 iterations of the Jacobi method for the
    following linear systems, using \(x^{(0)} = 0\).
    \begin{enumerate}
      \item[a.] \[
          \begin{array}{l}
            3x_1 - x_2 + x_3 = 1, \\
            3x_1 + 6x_2 + 2x_3 = 0, \\
            3x_1 + 3x_2 + 7x_3 = 4.
          \end{array}
        \]

        \underline{Sol}:\\
        For the Jacobi method, solve each equation for \(x_1, x_2, x_3\):
        \[
          \begin{array}{l}
            x_1^{(k+1)} = \frac{1 + x_2^{(k)} - x_3^{(k)}}{3}, \\
            x_2^{(k+1)} = \frac{-3x_1^{(k)} - 2x_3^{(k)}}{6}, \\
            x_3^{(k+1)} = \frac{4 - 3x_1^{(k)} - 3x_2^{(k)}}{7}.
          \end{array}
        \]
        Starting with \(x^{(0)} = [0, 0, 0]\):
        \[
          \begin{array}{l}
            \text{Iteration 1:} \\
            x_1^{(1)} = \frac{1 + 0 - 0}{3} = \frac{1}{3}, \\
            x_2^{(1)} = \frac{-3(0) - 2(0)}{6} = 0, \\
            x_3^{(1)} = \frac{4 - 0 - 0}{7} = \frac{4}{7}. \\
            \\
            \text{Iteration 2:} \\
            x_1^{(2)} = \frac{1 + 0 - \frac{4}{7}}{3} =
            \frac{\frac{3}{7}}{3} = \frac{1}{7}, \\
            x_2^{(2)} = \frac{-3\left(\frac{1}{3}\right) -
            2\left(\frac{4}{7}\right)}{6} = \frac{-1 -
            \frac{8}{7}}{6} = -\frac{5}{14}, \\
            x_3^{(2)} = \frac{4 - 3\left(\frac{1}{3}\right) - 0}{7} =
            \frac{3}{7}.
          \end{array}
        \]
        After two iterations: \(\boxed{\left[ \dfrac{1}{7},
        -\dfrac{5}{14}, \dfrac{3}{7} \right]}\).
    \end{enumerate}

  \item[2.] Find the first 2 iterations of the Jacobi method for the
    following linear systems, using the Gauss-Seidel method.
    \begin{enumerate}
      \item[a.] \[
          \begin{array}{l}
            3x_1 - x_2 + x_3 = 1, \\
            3x_1 + 6x_2 + 2x_3 = 0, \\
            3x_1 + 3x_2 + 7x_3 = 4.
          \end{array}
        \]

        \underline{Sol}:\\
        For the Gauss-Seidel method, update variables sequentially
        using the latest values:
        \[
          \begin{array}{l}
            x_1^{(k+1)} = \frac{1 + x_2^{(k)} - x_3^{(k)}}{3}, \\
            x_2^{(k+1)} = \frac{-3x_1^{(k+1)} - 2x_3^{(k)}}{6}, \\
            x_3^{(k+1)} = \frac{4 - 3x_1^{(k+1)} - 3x_2^{(k+1)}}{7}.
          \end{array}
        \]
        Starting with \(x^{(0)} = [0, 0, 0]\):
        \[
          \begin{array}{l}
            \text{Iteration 1:} \\
            x_1^{(1)} = \frac{1 + 0 - 0}{3} = \frac{1}{3}, \\
            x_2^{(1)} = \frac{-3\left(\frac{1}{3}\right) - 2(0)}{6} =
            -\frac{1}{6}, \\
            x_3^{(1)} = \frac{4 - 3\left(\frac{1}{3}\right) -
            3\left(-\frac{1}{6}\right)}{7} = \frac{13}{21}. \\
            \\
            \text{Iteration 2:} \\
            x_1^{(2)} = \frac{1 + \left(-\frac{1}{6}\right) -
            \frac{13}{21}}{3} = \frac{1}{9}, \\
            x_2^{(2)} = \frac{-3\left(\frac{1}{9}\right) -
            2\left(\frac{13}{21}\right)}{6} = -\frac{2}{9}, \\
            x_3^{(2)} = \frac{4 - 3\left(\frac{1}{9}\right) -
            3\left(-\frac{2}{9}\right)}{7} = \frac{13}{21}.
          \end{array}
        \]
        After two iterations: \(\boxed{\left[ \dfrac{1}{9},
        -\dfrac{2}{9}, \dfrac{13}{21} \right]}\).
    \end{enumerate}

  \item[4.] Use the Gauss-Seidel method to solve the linear system
    with \(TOL = 10^{-3}\) in the \(l_{\infty}\) norm.
    \begin{enumerate}
      \item[a.] \[
          \begin{array}{l}
            3x_1 - x_2 + x_3 = 1, \\
            3x_1 + 6x_2 + 2x_3 = 0, \\
            3x_1 + 3x_2 + 7x_3 = 4.
          \end{array}
        \]
    \end{enumerate}

    \underline{Sol}:\\
    Apply the Gauss-Seidel method with \(TOL = 10^{-3}\) in the
    \(l_\infty\) norm. Update equations:
    \[
      \begin{array}{l}
        x_1^{(k+1)} = \dfrac{1 + x_2^{(k)} - x_3^{(k)}}{3}, \\
        x_2^{(k+1)} = \dfrac{-3x_1^{(k+1)} - 2x_3^{(k)}}{6}, \\
        x_3^{(k+1)} = \dfrac{4 - 3x_1^{(k+1)} - 3x_2^{(k+1)}}{7}.
      \end{array}
    \]
    Starting with \(x^{(0)} = [0, 0, 0]\):
    \[
      \begin{array}{l}
        \text{Iteration 1:} \\
        x_1^{(1)} = \dfrac{1}{3} \approx 0.3333, \\
        x_2^{(1)} = -\dfrac{1}{6} \approx -0.1667, \\
        x_3^{(1)} = \dfrac{13}{21} \approx 0.6190. \\
        \text{Difference: } \max(|0.3333|, |0.1667|, |0.6190|) =
        0.6190 > 10^{-3}. \\
        \\
        \text{Iteration 2:} \\
        x_1^{(2)} = \dfrac{1}{9} \approx 0.1111, \\
        x_2^{(2)} = -\dfrac{2}{9} \approx -0.2222, \\
        x_3^{(2)} = \dfrac{13}{21} \approx 0.6190. \\
        \text{Difference: } \max(0.2222, 0.0555, 0.0000) = 0.2222 > 10^{-3}. \\
        \\
        \text{Iteration 3:} \\
        x_1^{(3)} \approx 0.0529, \\
        x_2^{(3)} \approx -0.2328, \\
        x_3^{(3)} \approx 0.6485. \\
        \text{Difference: } \max(0.0582, 0.0106, 0.0295) = 0.0582 > 10^{-3}. \\
        \\
        \text{Iteration 4:} \\
        x_1^{(4)} \approx 0.0396, \\
        x_2^{(4)} \approx -0.2360, \\
        x_3^{(4)} \approx 0.6556. \\
        \text{Difference: } \max(0.0133, 0.0032, 0.0071) = 0.0133 > 10^{-3}. \\
        \\
        \text{Iteration 5:} \\
        x_1^{(5)} \approx 0.0361, \\
        x_2^{(5)} \approx -0.2366, \\
        x_3^{(5)} \approx 0.6574. \\
        \text{Difference: } \max(0.0035, 0.0006, 0.0018) = 0.0035 > 10^{-3}. \\
        \\
        \text{Iteration 6:} \\
        x_1^{(6)} \approx 0.0353, \\
        x_2^{(6)} \approx -0.2368, \\
        x_3^{(6)} \approx 0.6578. \\
        \text{Difference: } \max(0.0008, 0.0002, 0.0004) = 0.0008 < 10^{-3}. \\
      \end{array}
    \]
    Convergence achieved after 6 iterations. Solution:
    \(\boxed{[0.0353, -0.2368, 0.6578]}\).

  \item[5.] The linear system
    \[
      \begin{aligned}
        x_1 - x_3 &= 0.2, \\
        - \frac{1}{2} x_1 + x_2 - \frac{1}{4} x_3 &= -1.425, \\
        x_1 - \frac{1}{2} x_2 + x_3 &= 2
      \end{aligned}
    \]

    has the solution \((0.9, -0.8, 0.7)^t\).

    \underline{Sol}:\\
    \textbf{a.} Check strict diagonal dominance for \(A\):
    \[
      \begin{array}{l}
        \text{Row 1: } |1| \not> |0| + |-1| =1 \quad (\text{Not dominant}), \\
        \text{Row 2: } |1| > \left|-\dfrac{1}{2}\right| +
        \left|-\dfrac{1}{4}\right| = \dfrac{3}{4} \quad (\text{Dominant}), \\
        \text{Row 3: } |1| \not> |1| + \left|-\dfrac{1}{2}\right| =
        \dfrac{3}{2} \quad (\text{Not dominant}). \\
      \end{array}
    \]
    Matrix \(A\) is \textbf{not} strictly diagonally dominant.

    \textbf{b.} Compute spectral radius of Jacobi matrix \(T_j\).
    Split \(A = D - L - U\), then \(T_j = D^{-1}(L + U)\):
    \[
      T_j =
      \begin{bmatrix}
        0 & 0 & -1 \\
        -\dfrac{1}{2} & 0 & -\dfrac{1}{4} \\
        1 & -\dfrac{1}{2} & 0
      \end{bmatrix}.
    \]
    Characteristic equation: \(\det(T_j - \lambda I) = -\lambda^3 -
    0.875\lambda - 0.25 = 0\). Solving numerically:
    \[
      \lambda_1 \approx -0.2646, \quad \lambda_{2,3} \approx 0.1323
      \pm 0.962i \quad (\text{modulus } \approx 0.971).
    \]
    Spectral radius \(\rho(T_j) \approx \boxed{0.971}\).

    \textbf{c.} Jacobi iterations (starting at \(x^{(0)} = 0\)):
    \[
      \begin{array}{l}
        x_1^{(k+1)} = 0.2 + x_3^{(k)}, \\
        x_2^{(k+1)} = -1.425 + 0.5x_1^{(k)} + 0.25x_3^{(k)}, \\
        x_3^{(k+1)} = 2 - x_1^{(k)} + 0.5x_2^{(k)}. \\
      \end{array}
    \]
    Iterations converge slowly (spectral radius \(\approx\) 0.971).
    After \(\approx\) 150 iterations (within 300), tolerance
    \(10^{-2}\) is achieved.
    \textbf{d.} Modified system:
    \[
      \begin{aligned}
        x_1 - 2x_3 &= 0.2, \\
        -0.5x_1 + x_2 -0.25x_3 &= -1.425, \\
        x_1 -0.5x_2 + x_3 &= 2.
      \end{aligned}
    \]
    New \(A_{\text{new}} =
      \begin{bmatrix}
        1 & 0 & -2 \\
        -0.5 & 1 & -0.25 \\
        1 & -0.5 & 1
    \end{bmatrix}\). Jacobi matrix \(T_j^{\text{new}}\) has
    eigenvalues with spectral radius \(\approx 1.394 > 1\). Thus,
    Jacobi method \textbf{diverges}.
\end{enumerate}

\subsection*{7.6 Error Bounds and Iterative Refinement}

1a, 1e, 2e, 3

\begin{enumerate}
  \item[1.] Compute the \(l_{infty}\) condition numbers of the
    following matrices.
    \begin{enumerate}
      \item[a.] \[
          \begin{bmatrix}
            \frac{1}{2} & \frac{1}{3} \\
            \frac{1}{3} & \frac{1}{4}
          \end{bmatrix}
        \]
      \item[e.] \[
          \begin{bmatrix}
            1 & -1 & -1 \\
            0 & 1 & -1 \\
            0 & 0 & -1
          \end{bmatrix}
        \]
    \end{enumerate}

    \underline{Sol}:\\
    \textbf{a.} Compute \(\kappa_\infty(A)\) for \(
      \begin{bmatrix} \frac{1}{2} & \frac{1}{3} \\ \frac{1}{3} & \frac{1}{4}
    \end{bmatrix}\):
    \[
      \begin{array}{l}
        \|A\|_\infty = \max\left(\frac{1}{2} + \frac{1}{3},
        \frac{1}{3} + \frac{1}{4}\right) = \frac{5}{6},  \\
        A^{-1} = 72
        \begin{bmatrix} \frac{1}{4} & -\frac{1}{3} \\ -\frac{1}{3} &
          \frac{1}{2}
        \end{bmatrix} =
        \begin{bmatrix} 18 & -24 \\ -24 & 36
        \end{bmatrix},
        \|A^{-1}\|_\infty = \max(42, 60) = 60,
        \kappa_\infty(A) = \frac{5}{6} \times 60 = \boxed{50}.
      \end{array}
    \]

    \textbf{e.} Compute \(\kappa_\infty(A)\) for \(
      \begin{bmatrix} 1 & -1 & -1 \\ 0 & 1 & -1 \\ 0 & 0 & -1
    \end{bmatrix}\):
    \[
      \|A\|_\infty = \max(3, 2, 1) = 3,
      A^{-1} =
      \begin{bmatrix} 1 & 1 & -2 \\ 0 & 1 & -1 \\ 0 & 0 & -1
      \end{bmatrix},
      \|A^{-1}\|_\infty = \max(4, 2, 1) = 4,
      \kappa_\infty(A) = 3 \times 4 = \boxed{12}.
    \]

  \item[2.] The following linear system \(Ax = b\) has \(x\) as the
    actual solution and \(\tilde{x}\) as an approximate solution.
    Using the results of the previous exercise, compute \(||x -
    \tilde{x}||_{\infty}\) and
    \[
      K_{\infty} (A) = \frac{||b - A \tilde{x}||_{\infty}}{||A||_{\infty}}
    \]

    \begin{enumerate}
      \item[e.] \[
          \begin{aligned}
            x_1 - x_2 - x_3 &= 2 \pi, \\
            x_2 - x_3 &= 0, \\
            -x_3 &= \pi, \\
          \end{aligned}
        \]
        \(x = (0, - \pi, - \pi )^t, \tilde{x} = (-0.1, -3.15, -3.14)^t.\)
    \end{enumerate}

    \underline{Sol}:\\
    Compute \(\|x - \tilde{x}\|_\infty\):
    \[
      x - \tilde{x} =
      \begin{bmatrix} 0 - (-0.1) \\ -\pi - (-3.15) \\ -\pi - (-3.14)
      \end{bmatrix} \approx
      \begin{bmatrix} 0.1 \\ 0.0084 \\ 0.0016
      \end{bmatrix}, \quad \|x - \tilde{x}\|_\infty = \boxed{0.1}.
    \]

    Compute \(K_\infty(A) = \frac{\|b - A\tilde{x}\|_\infty}{\|A\|_\infty}\):
    Residual \(b - A\tilde{x}\):
    \[
      \begin{bmatrix} 2\pi - (-0.1 - (-3.15) - (-3.14) \\ 0 - (-3.15
          - (-3.14)) \\ \pi - (-(-3.14))
        \end{bmatrix} \approx
        \begin{bmatrix} 0.0932 \\ 0.01 \\ 0.0016
        \end{bmatrix}, \quad \|b - A\tilde{x}\|_\infty \approx 0.0932.
      \]
      From Exercise 1(e), \(\|A\|_\infty = 3\):
      \[
        K_\infty(A) = \frac{0.0932}{3} \approx \boxed{0.0311}.
      \]

    \item[3.] The linear system
      \[
        \begin{bmatrix}
          1 & 2 \\
          1.0001 & 2
        \end{bmatrix}
        \begin{bmatrix}
          x_1 \\
          x_2
        \end{bmatrix}
        =
        \begin{bmatrix}
          3 \\
          3.0001
        \end{bmatrix}
      \]

      has the solution \((1, 1)^t\). Change \(A\) slightly to

      \[
        \begin{bmatrix}
          1 & 2 \\
          0.9999 & 2
        \end{bmatrix}
      \]

      and consider the linear system

      \[
        \begin{bmatrix}
          1 & 2 \\
          0.9999 & 2
        \end{bmatrix}
        \begin{bmatrix}
          x_1 \\
          x_2
        \end{bmatrix}
        =
        \begin{bmatrix}
          3 \\
          3.0001
        \end{bmatrix}.
      \]

      Compute the new solution using five-digit rounding arithmetic,
      and compare the change in A to the change in x.

      \underline{Sol}:\\
      Solve perturbed system \(
        \begin{bmatrix} 1 & 2 \\ 0.9999 & 2
        \end{bmatrix}
        \begin{bmatrix} x_1 \\ x_2
        \end{bmatrix} =
        \begin{bmatrix} 3 \\ 3.0001
      \end{bmatrix}\):
      \[
        \begin{array}{l}
          1.\ x_1 = 3 - 2x_2 \\
          2.\ 0.9999(3 - 2x_2) + 2x_2 = 3.0001 \\
          \Rightarrow 2.9997 - 1.9998x_2 + 2x_2 = 3.0001 \\
          \Rightarrow 0.0002x_2 = 0.0004 \Rightarrow x_2 = 2 \\
          3.\ x_1 = 3 - 2(2) = -1 \\
        \end{array}
      \]
      New solution: \(\tilde{x} =
        \begin{bmatrix} -1 \\ 2
      \end{bmatrix}\).

      Compare perturbations:
      \[
        \begin{aligned}
          \frac{\|A_{\text{new}} - A\|_\infty}{\|A\|_\infty} &=
          \frac{|0.9999 - 1.0001|}{\max(3, 3.0001)} =
          \frac{0.0002}{3} \approx 0.0067\%, \\
          \frac{\|x - \tilde{x}\|_\infty}{\|x\|_\infty} &=
          \frac{\max(|1 - (-1)|, |1 - 2|)}{1} = \frac{2}{1} = 200\%.
        \end{aligned}
      \]
      Result: \(\boxed{(-1, 2)}\).
  \end{enumerate}
