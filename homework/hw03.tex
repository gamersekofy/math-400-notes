\begin{center}
  \section*{Homework 03 - 2.4, 2.3, 2.5}
  Due Tue 2/18 \\
  Uzair Hamed Mohammed
\end{center}

\subsection*{2.4 Newton's Methods}

2, 4, 5, 7a, 9, 11, 12

\begin{enumerate}
  \item[2.] Let \( f(x) = -x^3 - \cos x \) and \( p_0 = -1 \). Use
    Newton's method to find \( p_2 \). Could \( p_0 = 0 \) be used
    for this problem?

    \underline{Sol}:\\
    \[
      \begin{array}{l}
        f(x) = -x^3 - \cos x \\
        f'(x) = -3x^2 + \sin x \\
        p_{n+1} = p_n - \frac{f(p_n)}{f'(p_n)} \\
        p_0 = -1 \\
        f(-1) = 1 - \cos(1) \\
        f'(-1) = -3 - \sin(1) \\
        p_1 = -1 - \frac{1 - \cos(1)}{-3 - \sin(1)} = -1 + \frac{1 -
        \cos(1)}{3 + \sin(1)} \approx -1 + \frac{1 - 0.5403}{3 +
        0.8415} \approx -0.8803 \\
        f(p_1) = f(-0.8803) = -(-0.8803)^3 - \cos(-0.8803) \approx 0.0453 \\
        f'(p_1) = f'(-0.8803) = -3(-0.8803)^2 + \sin(-0.8803) \approx -3.0961 \\
        p_2 = p_1 - \frac{f(p_1)}{f'(p_1)} \approx -0.8803 -
        \frac{0.0453}{-3.0961} \approx -0.8657 \\
        f'(0) = -3(0)^2 + \sin(0) = 0
      \end{array}
    \]

    $\boxed{p_2 \approx -0.8657, \text{ No, } p_0 = 0 \text{ because
    } f'(0) = 0}$

  \item[4.] Use Newton's method to find solutions accurate to within
    \( 10^{-5} \) for the following problems.
    \begin{enumerate}
      \item[a.] \( 2x \cos 2x - (x - 2)^2 = 0 \), on [2, 3] and [3, 4]

        \underline{Sol}:\\
        For part a, \( f(x) = 2x \cos 2x - (x - 2)^2 \), \( f'(x) = 2
        \cos 2x - 4x \sin 2x - 2(x - 2) \)

        Interval [2, 3], \( p_0 = 2.435 \):
        \[
          \begin{array}{l}
            p_0 = 2.435 \\
            f(p_0) = -0.211617 \\
            f'(p_0) = 8.859762 \\
            p_1 = p_0 - \frac{f(p_0)}{f'(p_0)} \approx 2.458918 \\
            p_2 = 2.458918 - \frac{f(2.458918)}{f'(2.458918)} \approx
            2.418642 \\
            p_3 = 2.418642 - \frac{f(2.418642)}{f'(2.418642)} \approx
            2.464706 \\
            p_4 = 2.464706 - \frac{f(2.464706)}{f'(2.464706)} \approx
            2.414600 \\
          \end{array}
        \]
        Restart with \( p_0 = 2.435 \):
        \[
          \begin{array}{l}
            p_0 = 2.435 \\
            p_1 = 2.43543449 \\
            p_2 = 2.43543445 \\
          \end{array}
        \]
        Root in [2, 3]: \( \boxed{2.43543} \)

        Interval [3, 4], \( p_0 = 3.877 \):
        \[
          \begin{array}{l}
            p_0 = 3.877 \\
            f(p_0) = 0.036466 \\
            f'(p_0) = -18.52455 \\
            p_1 = 3.877 - \frac{f(p_0)}{f'(p_0)} \approx 3.877597 \\
            p_2 = 3.877597 - \frac{f(3.877597)}{f'(3.877597)} \approx
            3.877570 \\
            p_3 = 3.877570 - \frac{f(3.877570)}{f'(3.877570)} \approx
            3.877570 \\
          \end{array}
        \]
        Root in [3, 4]: \( \boxed{3.87757} \)

      \item[b.] \( (x - 2)^2 - \ln x = 0 \), on [1, 2] and [e, 4]

        \underline{Sol}:\\
        For part b, \( f(x) = (x - 2)^2 - \ln x \), \( f'(x) = 2(x -
        2) - \frac{1}{x} \)

        Interval [1, 2], \( p_0 = 1.5 \):
        \[
          \begin{array}{l}
            p_0 = 1.5 \\
            f(p_0) = 0.09453489 \\
            f'(p_0) = -0.33333333 \\
            p_1 = p_0 - \frac{f(p_0)}{f'(p_0)} \approx 1.7831098 \\
            |p_1 - p_0| \approx 0.2831098 \\
            p_2 = p_1 - \frac{f(p_1)}{f'(p_1)} \\
            f(p_1) = f(1.7831098) \approx -0.052035 \\
            f'(p_1) = f'(1.7831098) \approx 0.442325 \\
            p_2 \approx 1.7831098 - \frac{-0.052035}{0.442325}
            \approx 1.899093 \\
            |p_2 - p_1| \approx 0.115983 \\
            p_3 = p_2 - \frac{f(p_2)}{f'(p_2)} \\
            f(p_2) = f(1.899093) \approx 0.002553 \\
            f'(p_2) = f'(1.899093) \approx 0.736535 \\
            p_3 \approx 1.899093 - \frac{0.002553}{0.736535} \approx 1.895623 \\
            |p_3 - p_2| \approx 0.003470 \\
            p_4 = p_3 - \frac{f(p_3)}{f'(p_3)} \\
            f(p_3) = f(1.895623) \approx 0.000006 \\
            f'(p_3) = f'(1.895623) \approx 0.726156 \\
            p_4 \approx 1.895623 - \frac{0.000006}{0.726156} \approx 1.895615 \\
            |p_4 - p_3| \approx 0.000008 \\
            p_5 = 1.895615 - \frac{f(1.895615)}{f'(1.895615)} \approx
            1.895615 \\
            |p_5 - p_4| \approx 0.000000 \\
          \end{array}
        \]
        Root in [1, 2]: \( \boxed{1.89562} \)

        Interval [e, 4], \( p_0 = 3 \):
        \[
          \begin{array}{l}
            p_0 = 3 \\
            f(p_0) = 0.9013877 \\
            f'(p_0) = 1.6666666 \\
            p_1 = p_0 - \frac{f(p_0)}{f'(p_0)} \approx 2.458134 \\
            |p_1 - p_0| \approx 0.541866 \\
            p_2 = p_1 - \frac{f(p_1)}{f'(p_1)} \\
            f(p_1) = f(2.458134) \approx -0.248548 \\
            f'(p_1) = f'(2.458134) \approx 0.911264 \\
            p_2 \approx 2.458134 - \frac{-0.248548}{0.911264} \approx
            2.730853 \\
            |p_2 - p_1| \approx 0.272719 \\
            p_3 = p_2 - \frac{f(p_2)}{f'(p_2)} \\
            f(p_2) = f(2.730853) \approx -0.018187 \\
            f'(p_2) = f'(2.730853) \approx 1.43225 \\
            p_3 \approx 2.730853 - \frac{-0.018187}{1.43225} \approx 2.743549 \\
            |p_3 - p_2| \approx 0.012696 \\
            p_4 = p_3 - \frac{f(p_3)}{f'(p_3)} \\
            f(p_3) = f(2.743549) \approx -0.000115 \\
            f'(p_3) = f'(2.743549) \approx 1.45855 \\
            p_4 \approx 2.743549 - \frac{-0.000115}{1.45855} \approx 2.743628 \\
            |p_4 - p_3| \approx 0.000079 \\
            p_5 = p_4 - \frac{f(p_4)}{f'(p_4)} \\
            f(p_4) = f(2.743628) \approx -0.00000004 \\
            f'(p_4) = f'(2.743628) \approx 1.45871 \\
            p_5 \approx 2.743628 - \frac{-0.00000004}{1.45871}
            \approx 2.743628 \\
            |p_5 - p_4| \approx 0.000000 \\
          \end{array}
        \]
        Root in [e, 4]: \( \boxed{2.74363} \)

      \item[c.] \( e^x -3x^2 = 0 \), on [0, 1] and [3, 5]

        \underline{Sol}:\\
        For part c, \( f(x) = e^x - 3x^2 \), \( f'(x) = e^x - 6x \)

        Interval [0, 1], \( p_0 = 0.5 \):
        \[
          \begin{array}{l}
            p_0 = 0.5 \\
            p_1 = 0.683939 \\
            p_2 = 0.697418 \\
            p_3 = 0.6975 \\
          \end{array}
        \]
        Root in [0, 1]: \( \boxed{0.6975} \)

        Interval [3, 5], \( p_0 = 3 \):
        \[
          \begin{array}{l}
            p_0 = 3 \\
            p_1 = 2.7666 \\
            p_2 = 2.7456 \\
            p_3 = 2.7454 \\
          \end{array}
        \]
        Root in [3, 5]: \( \boxed{2.7454} \)

      \item[d.] \( \sin x - e^{-x} = 0 \), on [0, 1], [3, 4], and [6, 7]

        \underline{Sol}:\\
        For part d, \( f(x) = \sin x - e^{-x} \), \( f'(x) = \cos x + e^{-x} \)

        Interval [0, 1], \( p_0 = 0 \):
        \[
          \begin{array}{l}
            p_0 = 0 \\
            p_1 = 0.5 \\
            p_2 = 0.58612 \\
            p_3 = 0.58853 \\
            p_4 = 0.58853 \\
          \end{array}
        \]
        Root in [0, 1]: \( \boxed{0.58853} \)

        Interval [3, 4], \( p_0 = 3 \):
        \[
          \begin{array}{l}
            p_0 = 3 \\
            p_1 = 3.0993 \\
            p_2 = 3.0964 \\
            p_3 = 3.0964 \\
          \end{array}
        \]
        Root in [3, 4]: \( \boxed{3.0964} \)

        Interval [6, 7], \( p_0 = 6 \):
        \[
          \begin{array}{l}
            p_0 = 6 \\
            p_1 = 6.2857 \\
            p_2 = 6.2832 \\
            p_3 = 6.2832 \\
          \end{array}
        \]
        Root in [6, 7]: \( \boxed{6.2832} \)

    \end{enumerate}
  \item[5.] Use Newton's method to find all four solutions of \( 4x
    \cos (2x) - (x - 2)^2 = 0 \) in [0, 8] accurate to within \( 10^{-5} \)

    \underline{Sol}:\\
    Let \( f(x) = 4x \cos (2x) - (x - 2)^2 \) and \( f'(x) = 4 \cos
    (2x) - 8x \sin (2x) - 2(x - 2) \).

    For root around 2.36, \( p_0 = 1.5 \):
    \[
      \begin{array}{l}
        p_0 = 1.5 \\
        p_1 = 0.1698 \\
        p_2 = 1.433 \\
        p_3 = 2.155 \\
        p_4 = 2.355 \\
        p_5 = 2.36315 \\
        p_6 = 2.36317
      \end{array}
    \]
    Root 1: \( \boxed{2.36317} \)

    For root around 3.81, \( p_0 = 3.5 \):
    \[
      \begin{array}{l}
        p_0 = 3.5 \\
        p_1 = 3.8233 \\
        p_2 = 3.81793 \\
        p_3 = 3.81793 \\
      \end{array}
    \]
    Root 2: \( \boxed{3.81793} \)

    For root around 5.83, \( p_0 = 5.5 \):
    \[
      \begin{array}{l}
        p_0 = 5.5 \\
        p_1 = 5.8414 \\
        p_2 = 5.83925 \\
        p_3 = 5.83925 \\
      \end{array}
    \]
    Root 3: \( \boxed{5.83925} \)

    For root around 6.60, \( p_0 = 7 \):
    \[
      \begin{array}{l}
        p_0 = 7 \\
        p_1 = 6.6115 \\
        p_2 = 6.60309 \\
        p_3 = 6.60308 \\
      \end{array}
    \]
    Root 4: \( \boxed{6.60308} \)

  \item[7.] Use Newton's method to approximate the solutions of the
    following equations to within \( 10^{-5} \) in the given
    intervals. In these problems, the convergence will be slower than
    normal because the zeroes are not simple.
    \begin{enumerate}
      \item[a.] \( x^2 - 2xe^{-x} + e^{-2x} = 0 \), on [0, 1]

        \underline{Sol}:\\
        For \( f(x) = x^2 - 2xe^{-x} + e^{-2x} \), \( f'(x) = 2x +
        2xe^{-x} - 2e^{-x} - 2e^{-2x} \).
        Simplified Newton iteration formula: \( p_{n+1} = p_n -
        \frac{p_n - e^{-p_n}}{2 (1 + e^{-p_n})} \)

        Interval [0, 1], \( p_0 = 0.5 \):
        \[
          \begin{array}{l}
            p_0 = 0.5 \\
            p_1 = 0.533156 \\
            p_2 = 0.564948 \\
            p_3 = 0.567128 \\
            p_4 = 0.567135 \\
            p_5 = 0.567135 \\
            p_6 = 0.567135 \\
            p_7 = 0.567135 \\
            p_8 = 0.567135 \\
            p_9 = 0.567135 \\
            p_{10} = 0.567135 \\
            p_{11} = 0.567135 \\
            p_{12} = 0.567135 \\
            p_{13} = 0.567135 \\
          \end{array}
        \]
        Root in [0, 1]: \( \boxed{0.567135} \)

    \end{enumerate}

  \item[9.] Use Newton's method to find an approximation to \(
    \sqrt{3} \) correct to within \( 10^{-4} \), and compare the
    results to those obtained in Exercise 9 of Sections 2.2 and 2.3.

    \underline{Sol}:\\
    Let \( f(x) = x^2 - 3 \), \( f'(x) = 2x \).
    Newton's method iteration: \( p_{n+1} = p_n -
      \frac{f(p_n)}{f'(p_n)} = p_n - \frac{p_n^2 - 3}{2p_n} =
    \frac{1}{2} \left( p_n + \frac{3}{p_n} \right) \).
    Start with \( p_0 = 1.7 \).
    \[
      \begin{array}{l}
        p_0 = 1.7 \\
        p_1 = \frac{1}{2} \left( 1.7 + \frac{3}{1.7} \right) \approx
        1.73235294 \\
        |p_1 - p_0| \approx 0.03235 \\
        p_2 = \frac{1}{2} \left( p_1 + \frac{3}{p_1} \right) \approx
        1.73205081 \\
        |p_2 - p_1| \approx 0.000302 \\
        p_3 = \frac{1}{2} \left( p_2 + \frac{3}{p_2} \right) \approx
        1.73205081 \\
        |p_3 - p_2| \approx 0 \\
      \end{array}
    \]
    We need accuracy within \( 10^{-4} \), so check \( |p_2 - p_1|
    \approx 0.000302 > 10^{-4} \). Need more iterations.
    Let's recalculate with higher precision.
    \[
      \begin{array}{l}
        p_0 = 1.7 \\
        p_1 = 1.7323529411764706 \\
        p_2 = 1.7320508100147275 \\
        p_3 = 1.7320508075688772 \\
        |p_1 - p_0| \approx 0.03235 \\
        |p_2 - p_1| \approx 0.000302 \\
        |p_3 - p_2| \approx 2.445 \times 10^{-9} < 10^{-4} \\
      \end{array}
    \]
    So \( p_3 \approx 1.7320508 \) is accurate within \( 10^{-4} \)
    in 3 iterations.  We need to check if \( |p_2 - p_1| < 10^{-4}
    \). \( |p_2 - p_1| \approx 0.000302 > 10^{-4} \). So we need \(
    p_3 \).  Approximation is \( p_3 \approx 1.73205 \).

    Comparison to Exercise 9 of Sections 2.2 and 2.3:
    Bisection method on [1, 2] to get accuracy \( 10^{-4} \) requires
    \( n \geq \log_2 \left( \frac{2-1}{10^{-4}} \right) =
    \log_2(10^4) \approx 14 \) iterations.
    Newton's method requires only 3 iterations. Newton's method
    converges much faster than bisection method. False position
    method is also expected to be slower than Newton's method.

    Approximation to \( \sqrt{3} \) using Newton's method: \(
    \boxed{1.73205} \) in 3 iterations.

  \item[11.] Newton's method applied to the function \( f(x) = x^2 -2
    \) with a positive initial approximation \( p_0 \) converges to
    the only positive solution, \( \sqrt{2} \).
    \begin{enumerate}
      \item[a.] Show that Newton's method in this situation assumes
        the form that the Babylonians used to approximate \( \sqrt{2} \):
        \[
          p_{n + 1} = \frac{1}{2} p_n + \frac{1}{p_n}
        \]

        \underline{Sol}:\\
        For part a, we have \( f(x) = x^2 - 2 \). Then \( f'(x) = 2x \).
        Newton's method is given by \( p_{n+1} = p_n -
        \frac{f(p_n)}{f'(p_n)} \).
        Substituting \( f(x) \) and \( f'(x) \), we get
        \[
          p_{n+1} = p_n - \frac{p_n^2 - 2}{2p_n}
        \]
        We can rewrite this as
        \[
          p_{n+1} = \frac{2p_n^2}{2p_n} - \frac{p_n^2 - 2}{2p_n} =
          \frac{2p_n^2 - (p_n^2 - 2)}{2p_n} = \frac{2p_n^2 - p_n^2 +
          2}{2p_n} = \frac{p_n^2 + 2}{2p_n}
        \]
        \[
          p_{n+1} = \frac{p_n^2}{2p_n} + \frac{2}{2p_n} =
          \frac{p_n}{2} + \frac{1}{p_n} = \frac{1}{2} p_n + \frac{1}{p_n}
        \]
        This is the Babylonian method for approximating \( \sqrt{2} \).

        \boxed{p_{n + 1} = \frac{1}{2} p_n + \frac{1}{p_n}}

      \item[b.] Use the sequence in (a) with \( p_0 = 1 \) to
        determine an approximation that is accurate to within \( 10^{-5} \)

        \underline{Sol}:\\
        For part b, we use the iterative formula \( p_{n+1} =
        \frac{1}{2} p_n + \frac{1}{p_n} \) with \( p_0 = 1 \).
        \[
          \begin{array}{l}
            p_0 = 1 \\
            p_1 = \frac{1}{2} p_0 + \frac{1}{p_0} = \frac{1}{2} (1) +
            \frac{1}{1} = 1.5 \\
            |p_1 - p_0| = |1.5 - 1| = 0.5 \\
            p_2 = \frac{1}{2} p_1 + \frac{1}{p_1} = \frac{1}{2} (1.5)
            + \frac{1}{1.5} = 1.4166 \\
            |p_2 - p_1| = |1.41666 - 1.5| \approx 0.08333 \\
            p_3 = \frac{1}{2} p_2 + \frac{1}{p_2} = \frac{1}{2}
            (1.4166) + \frac{1}{1.4166} \approx 1.41421 \\
            |p_3 - p_2| = |1.41421 - 1.4166| \approx 0.002451 \\
            p_4 = \frac{1}{2} p_3 + \frac{1}{p_3} = \frac{1}{2}
            (1.41421) + \frac{1}{1.4142} \approx 1.4142 \\
            |p_4 - p_3| = |1.41421 - 1.4142| \approx 2.1239 \times
            10^{-6} < 10^{-5} \\
          \end{array}
        \]
        Since \( |p_4 - p_3| < 10^{-5} \), we can take \( p_4 \) as
        the approximation.

        \boxed{1.41421}
    \end{enumerate}

  \item[12.] In Exersise 14 of Section 2.3, we found that for \( f(x)
    = \tan \pi x - 6 \), the Bisection method on [0, 0.48] converges
    more quickly than the method of False Position with \( p_0 = 0 \)
    and \( p_1 = 0.48 \). Also, the Secant method with these values
    of \( p_0 \) and \( p_1 \) does not give convergence. Apply
    Newton's method to this problem with (a) \( p_0 = 0 \) and (b) \(
    p_0 = 0.48 \). (c) Explain the reason for any discrepancies.

    \underline{Sol}:\\
    For \( f(x) = \tan(\pi x) - 6 \), \( f'(x) = \pi \sec^2(\pi x) \).
    Newton's method iteration: \( p_{n+1} = p_n - \frac{\tan(\pi p_n)
    - 6}{\pi \sec^2(\pi p_n)} \)

    (a) \( p_0 = 0 \):
    \[
      \begin{array}{l}
        p_0 = 0 \\
        p_1 = 0 - \frac{\tan(0) - 6}{\pi \sec^2(0)} = \frac{6}{\pi}
        \approx 1.90986 \\
      \end{array}
    \]
    Diverges immediately.

    (b) \( p_0 = 0.48 \):
    \[
      \begin{array}{l}
        p_0 = 0.48 \\
        p_1 \approx 0.482727 \\
        p_2 \approx 0.481454 \\
        p_3 \approx 0.48016 \\
        p_4 \approx 0.47887 \\
        p_5 \approx 0.47758 \\
        p_6 \approx 0.47629 \\
        p_7 \approx 0.47501 \\
        p_8 \approx 0.47373 \\
        p_9 \approx 0.47245 \\
        p_{10} \approx 0.47118 \\
        \vdots \\
        p_{90} \approx 0.448614 \\
        p_{91} \approx 0.448614 \\
      \end{array}
    \]
    Converges slowly to \( \approx 0.448614 \).

    (c) Explanation:
    For \( p_0 = 0 \), Newton's method diverges as \( p_1 =
    \frac{6}{\pi} \notin [0, 0.48] \).
    For \( p_0 = 0.48 \), Newton's method converges very slowly.
    Bisection method in Exercise 14 of Section 2.3 converged faster
    than False Position. Secant method diverged.
    Newton's method convergence depends on \( p_0 \) and \( f'(x) \).
    Large \( |f'(x)| \) can lead to slow convergence as correction
    term \( -f(p_n)/f'(p_n) \) becomes small.
    For \( f(x) = \tan(\pi x) - 6 \) in [0, 0.48], near \( x = 0.5
    \), \( f'(x) = \pi \sec^2(\pi x) \) is large, potentially slowing
    convergence even when starting at \( p_0 = 0.48 \). Bisection's
    consistent interval halving can be more efficient in this case
    than Newton's or False Position, and Secant is unstable due to
    derivative behavior and starting points.

    (a) \( p_0 = 0 \): Diverges.
    (b) \( p_0 = 0.48 \): Converges slowly to \( \boxed{0.44861} \)
    (approximately after 90 iterations).
    (c) Explained above.
\end{enumerate}

\subsection*{2.3 The Secant Method}

3a, 4a, 11, 13, 14, 15

\begin{enumerate}
  \item[3a.] Use the Secant method to find solutions accurate to
    within \( 10^{-4} \) for \( x^3 - 2x^2 - 5 = 0 \), on [1, 4].

    \underline{Sol}:\\
    Let \( f(x) = x^3 - 2x^2 - 5 \).
    Secant method iteration: \( p_{n+1} = p_n - \frac{f(p_n)(p_n -
    p_{n-1})}{f(p_n) - f(p_{n-1})} \)
    Start with \( p_0 = 2, p_1 = 4 \).
    \[
      \begin{array}{l}
        p_0 = 2, f(p_0) = -5 \\
        p_1 = 4, f(p_1) = 27 \\
        p_2 = 4 - \frac{f(4)(4-2)}{f(4) - f(2)} = 2.3125 \\
        f(p_2) = f(2.3125) = -3.33154 \\
        p_3 = 2.3125 - \frac{f(2.3125)(2.3125 - 4)}{f(2.3125) - f(4)}
        \approx 2.49784 \\
        f(p_3) = f(2.49784) \approx -1.8903 \\
        p_4 = 2.49784 - \frac{f(2.49784)(2.49784 -
        2.3125)}{f(2.49784) - f(2.3125)} \approx 2.74089 \\
        f(p_4) = f(2.74089) \approx 0.5792 \\
        p_5 = 2.74089 - \frac{f(2.74089)(2.74089 -
        2.49784)}{f(2.74089) - f(2.49784)} \approx 2.6839 \\
        f(p_5) = f(2.6839) \approx -0.1003 \\
        p_6 = 2.6839 - \frac{f(2.6839)(2.6839 - 2.74089)}{f(2.6839) -
        f(2.74089)} \approx 2.69231 \\
        f(p_6) = f(2.69231) \approx -0.0105 \\
        p_7 = 2.69231 - \frac{f(2.69231)(2.69231 -
        2.6839)}{f(2.69231) - f(2.6839)} \approx 2.69133 \\
        f(p_7) = f(2.69133) \approx -0.00011 \\
        p_8 = 2.69133 - \frac{f(2.69133)(2.69133 -
        2.69231)}{f(2.69133) - f(2.69231)} \approx 2.69132 \\
        |p_8 - p_7| \approx |2.69132 - 2.69133| = 0.00001 < 10^{-4}
      \end{array}
    \]
    Approximation accurate to within \( 10^{-4} \) is \( p_8 \).

    \boxed{2.69132}

  \item[4a.] Use the Secant method to find solutions accurate to
    within \( 10^{-5} \) for \( 2x \cos 2x - (x - 2)^2 = 0 \), on [2,
    3] and on [3, 4].

    \underline{Sol}:\\
    Let \( f(x) = 2x \cos 2x - (x - 2)^2 \).
    Secant method iteration: \( p_{n+1} = p_n - \frac{f(p_n)(p_n -
    p_{n-1})}{f(p_n) - f(p_{n-1})} \)

    Interval [2, 3], \( p_0 = 2, p_1 = 3 \):
    \[
      \begin{array}{l}
        p_0 = 2, f(p_0) \approx -2.6131 \\
        p_1 = 3, f(p_1) \approx 4.7603 \\
        p_2 \approx 2.3543 \\
        f(p_2) \approx -0.4873 \\
        p_3 \approx 2.4289 \\
        f(p_3) \approx -0.0915 \\
        p_4 \approx 2.4351 \\
        f(p_4) \approx -0.0053 \\
        p_5 \approx 2.4354 \\
        f(p_5) \approx -0.0001 \\
        p_6 \approx 2.43543 \\
        f(p_6) \approx -0.000002 \\
        p_7 \approx 2.43543 \\
      \end{array}
    \]
    Root in [2, 3]: \( \boxed{2.43543} \)

    Interval [3, 4], \( p_0 = 3, p_1 = 4 \):
    \[
      \begin{array}{l}
        p_0 = 3, f(p_0) \approx 4.7603 \\
        p_1 = 4, f(p_1) \approx -2.8863 \\
        p_2 \approx 3.6233 \\
        f(p_2) \approx 1.2253 \\
        p_3 \approx 3.8045 \\
        f(p_3) \approx 0.2095 \\
        p_4 \approx 3.8304 \\
        f(p_4) \approx 0.0176 \\
        p_5 \approx 3.8326 \\
        f(p_5) \approx 0.0008 \\
        p_6 \approx 3.83269 \\
        f(p_6) \approx 0.00003 \\
        p_7 \approx 3.83269 \\
      \end{array}
    \]
    Root in [3, 4]: \( \boxed{3.83269} \)

  \item[11.] Approximate, to within \( 10^{-4} \), the value of \( x
    \) that produces the point on the graph of \( y = x^2 \) that is
    closest to (1, 0). [\textit{Hint:} Minimize \([d(x)]^2\), where
    \(d(x)\) represents the distance from \( (x, x^2)\) to \( (1, 0) \).]

    \underline{Sol}:\\
    Let \( f(x) = [d(x)]^2 = (x - 1)^2 + x^4 = x^4 + x^2 - 2x + 1 \).
    Minimize \( f(x) \) by finding roots of \( f'(x) = 0 \).
    \( g(x) = f'(x) = 4x^3 + 2x - 2 \)
    \( g'(x) = 12x^2 + 2 \)
    Newton's method iteration: \( p_{n+1} = p_n -
    \frac{g(p_n)}{g'(p_n)} = p_n - \frac{4p_n^3 + 2p_n - 2}{12p_n^2 + 2} \)
    Start with \( p_0 = 0.6 \).
    \[
      \begin{array}{l}
        p_0 = 0.6 \\
        p_1 = 0.5898734 \\
        p_2 = 0.5897549 \\
        p_3 = 0.5897549 \\
      \end{array}
    \]
    Since \( |p_2 - p_1| \approx 0.0001185 < 10^{-4} \) is not
    satisfied, we need to check \( |p_3 - p_2| \).
    \( |p_3 - p_2| = |0.5897549 - 0.5897549| \approx 0 < 10^{-4} \).
    Let's calculate one more iteration to be safe.
    \[
      \begin{array}{l}
        p_0 = 0.6 \\
        p_1 = 0.5898734 \\
        p_2 = 0.5897549297 \\
        p_3 = 0.5897549165 \\
      \end{array}
    \]
    \( |p_3 - p_2| \approx 1.32 \times 10^{-8} < 10^{-4} \).
    Thus \( p_2 = 0.5897549 \) is accurate to within \( 10^{-4} \) if
    we round to 4 decimal places. \( p_2 \approx 0.5898 \).

    \boxed{0.58975}

  \item[13.] The fourth-degree polynomial \( f(x) = 230x^4 + 18x^3 +
    9x^2 - 221x -9 \) has two real zeros, one in [-1, 0] and the
    other in [0, 1]. Attempt to approximate these zeros to within \(
    10^{-6} \) using each method.
    \begin{enumerate}
      \item[a.] method of False Position

        \underline{Sol}:\\

        Interval [-1, 0]: \( a_0 = -1, b_0 = 0 \)
        \[
          \begin{array}{l|lll}
            n & a_n & b_n & p_n \\
            \hline
            0 & -1 & 0 & - \\
            1 & -1 & 0 & -0.020361 \\
            2 & -0.040233 & -0.020361 & -0.040645 \\
            3 & -0.040645 & -0.020361 & -0.040658 \\
            4 & -0.040658 & -0.020361 & -0.040659 \\
            5 & -0.040659 & -0.020361 & -0.040659 \\
          \end{array}
        \]
        Root in [-1, 0]: \( \boxed{-0.040659} \)

        Interval [0, 1]: \( a_0 = 0, b_0 = 1 \)
        \[
          \begin{array}{l|lll}
            n & a_n & b_n & p_n \\
            \hline
            0 & 0 & 1 & - \\
            1 & 0 & 1 & 0.25 \\
            2 & 0 & 0.25 & 0.254286 \\
            3 & 0 & 0.254286 & 0.254343 \\
            4 & 0 & 0.254343 & 0.254344 \\
          \end{array}
        \]
        Root in [0, 1]: \( \boxed{0.254344} \) (False Position stagnates)


      \item[b.] Secant method

        Interval [-1, 0]: \( p_0 = -1, p_1 = 0 \)
        \[
          \begin{array}{l|lll}
            n & p_{n-1} & p_n & p_{n+1} \\
            \hline
            0 & -1 & 0 & - \\
            1 & -1 & 0 & -0.020361 \\
            2 & 0 & -0.020361 & -0.040722 \\
            3 & -0.020361 & -0.040722 & -0.040659 \\
            4 & -0.040722 & -0.040659 & -0.040659 \\
            5 & -0.040659 & -0.040659 & -0.040659 \\
          \end{array}
        \]
        Root in [-1, 0]: \( \boxed{-0.040659} \)

        Interval [0, 1]: \( p_0 = 0, p_1 = 1 \)
        \[
          \begin{array}{l|lll}
            n & p_{n-1} & p_n & p_{n+1} \\
            \hline
            0 & 0 & 1 & - \\
            1 & 0 & 1 & 0.25 \\
            2 & 1 & 0.25 & 0.254286 \\
            3 & 0.25 & 0.254286 & 0.95933 \\
            4 & 0.254286 & 0.95933 & 0.97385 \\
            5 & 0.95933 & 0.97385 & 0.97455 \\
            6 & 0.97385 & 0.97455 & 0.97455 \\
          \end{array}
        \]
        Root in [0, 1]: \( \boxed{0.97455} \) (Secant converges)
    \end{enumerate}

  \item[14.] The function \( f(x) = \tan \pi x - 6 \) has a zero at
    \( (1/\pi) \arctan 6 \approx 0.447431543 \). Let \( p_0 = 0\) and
    \( p_1 = 0.48 \) and use 10 iterations of each of the following
    methods to approximate this root. Which method is most successful and why?
    \begin{enumerate}
      \item[a.] Bisection method
      \item[b.] method of False Position
      \item[c.] Secant method
    \end{enumerate}

    \underline{Sol}:\\
    For \( f(x) = \tan(\pi x) - 6 \), root \( \approx 0.447431543 \).
    \( p_0 = 0, p_1 = 0.48 \).

    Part a: Bisection method, interval \( [a_0, b_0] = [0, 0.48] \)
    \[
      \begin{array}{l|llll}
        n & a_n & b_n & p_n & f(p_n) \\
        \hline
        0 & 0 & 0.48 & - & - \\
        1 & 0 & 0.48 & 0.24 & -4.453 \\
        2 & 0.24 & 0.48 & 0.36 & -2.189 \\
        3 & 0.36 & 0.48 & 0.42 & -0.659 \\
        4 & 0.42 & 0.48 & 0.45 & 0.759 \\
        5 & 0.42 & 0.45 & 0.435 & -0.047 \\
        6 & 0.435 & 0.45 & 0.4425 & 0.354 \\
        7 & 0.435 & 0.4425 & 0.43875 & 0.152 \\
        8 & 0.435 & 0.43875 & 0.436875 & 0.052 \\
        9 & 0.435 & 0.436875 & 0.4359375 & 0.002 \\
        10 & 0.435 & 0.4359375 & 0.43546875 & -0.022 \\
      \end{array}
    \]
    \( p_{10} \approx 0.43546875 \)

    Part b: False Position method, \( p_0 = 0, p_1 = 0.48 \)
    \[
      \begin{array}{l|lll}
        n & p_{n-1} & p_n & p_{n+1} \\
        \hline
        0 & 0 & 0.48 & - \\
        1 & 0 & 0.48 & 0.091324 \\
        2 & 0.091324 & 0.48 & 0.16533 \\
        3 & 0.16533 & 0.48 & 0.22535 \\
        4 & 0.22535 & 0.48 & 0.27436 \\
        5 & 0.27436 & 0.48 & 0.31389 \\
        6 & 0.31389 & 0.48 & 0.34576 \\
        7 & 0.34576 & 0.48 & 0.37145 \\
        8 & 0.37145 & 0.48 & 0.39226 \\
        9 & 0.39226 & 0.48 & 0.4092 \\
        10 & 0.4092 & 0.48 & 0.4230 \\
      \end{array}
    \]
    \( p_{10} \approx 0.4230 \)

    Part c: Secant method, \( p_0 = 0, p_1 = 0.48 \)
    \[
      \begin{array}{l|lll}
        n & p_{n-1} & p_n & p_{n+1} \\
        \hline
        0 & 0 & 0.48 & - \\
        1 & 0 & 0.48 & 0.48283 \\
        2 & 0.48 & 0.48283 & 0.44585 \\
        3 & 0.48283 & 0.44585 & 0.44744 \\
        4 & 0.44585 & 0.44744 & 0.44743 \\
        5 & 0.44744 & 0.44743 & 0.44743 \\
        6 & 0.44743 & 0.44743 & 0.44743 \\
      \end{array}
    \]
    \( p_{10} \approx 0.44743 \) (converged in 4 iterations to given accuracy)

    Most successful: Secant method converges fastest.
    Bisection method is guaranteed to converge, but slow.
    False Position is slow due to one endpoint remaining fixed and
    slow change in interval.
    Secant method is most successful as it converges quickly to the
    root with given initial approximations, even though False
    Position should theoretically be faster than Bisection, in this
    case, due to function's behavior, False Position is quite slow.
    Secant method takes advantage of recent two approximations to
    find next, leading to faster convergence in this problem.

  \item[15.] The sum of two numbers is 20. If each number is added to
    its square root, the product of the two sums is 155.55. Determine
    the two numbers to within \( 10^{-4} \).

    \underline{Sol}:\\
    Let \( f(x) = (x + \sqrt{x})(20 - x + \sqrt{20 - x}) - 155.55 = 0 \)\\
    \( f'(x) = \left(1 + \frac{1}{2\sqrt{x}}\right)(20 - x + \sqrt{20
    - x}) + (x + \sqrt{x})\left(-1 - \frac{1}{2\sqrt{20 - x}}\right) \)\\
    Newton's method \( p_{n+1} = p_n - \frac{f(p_n)}{f'(p_n)} \), \(
    p_0 = 6.5 \):
    \[
      \begin{array}{l}
        p_0 = 6.5 \\
        p_1 \approx 6.5127 \\
        p_2 \approx 6.51466 \\
        p_3 \approx 6.514758 \\
      \end{array}
    \]
    Let \( x \approx 6.5148 \), \( y = 20 - x \approx 13.4852 \).

    Check: \( (6.5148 + \sqrt{6.5148})(13.4852 + \sqrt{13.4852})
    \approx 155.55 \)

    \boxed{x \approx 6.5148, y \approx 13.4852}

\end{enumerate}

\subsection*{2.5 Error Analysis and Accelerating Convergence}

1a, 2a, 2c, 3, 5.

\begin{enumerate}
  \item[1a.] This sequence is linearly convergent. Generate the first
    five terms of the sequence \( \lbrace q_n \rbrace \) using
    Aitken's \( \Delta^2 \) method: \( p_0 = 0.5, p_n = (2 - e^{p_n
    -1} + p^2_{n-1})/3 \), for \( n \ge 1 \).

    \underline{Sol}:\\
    Given \( p_0 = 0.5 \), \( p_n = (2 - e^{p_{n-1}} + p^2_{n-1})/3
    \) for \( n \ge 1 \).
    First six terms of \( \{ p_n \} \):
    \[
      \begin{array}{l}
        p_0 = 0.5 \\
        p_1 \approx 0.2004266667 \\
        p_2 \approx 0.2727492667 \\
        p_3 \approx 0.2535640667 \\
        p_4 \approx 0.2585616667 \\
        p_5 \approx 0.257262 \\
        p_6 \approx 0.2576003333 \\
      \end{array}
    \]
    Aitken's \( \Delta^2 \) method: \( q_n = p_n - \frac{(p_{n+1} -
    p_n)^2}{(p_{n+2} - 2p_{n+1} + p_n)} \)
    \[
      \begin{array}{l}
        q_0 \approx p_0 - \frac{(p_1 - p_0)^2}{(p_2 - 2p_1 + p_0)}
        \approx 0.25869 \\
        q_1 \approx p_1 - \frac{(p_2 - p_1)^2}{(p_3 - 2p_2 + p_1)}
        \approx 0.25760 \\
        q_2 \approx p_2 - \frac{(p_3 - p_2)^2}{(p_4 - 2p_3 + p_2)}
        \approx 0.25753 \\
        q_3 \approx p_3 - \frac{(p_4 - p_3)^2}{(p_5 - 2p_4 + p_3)}
        \approx 0.25753 \\
        q_4 \approx p_4 - \frac{(p_5 - p_4)^2}{(p_6 - 2p_5 + p_4)}
        \approx 0.25753 \\
      \end{array}
    \]

    \boxed{q_0 = 0.25869, q_1 = 0.25760, q_2 = 0.25753, q_3 =
    0.25753, q_4 = 0.25753}

  \item[2a.] Newton's method does not converge quadratically for
    these problems. Accelerate the convergence using Aitken's \(
    \Delta^2 \) method. Iterate until \( |q_n - q_{n-1}| < 10^{-4} \).
    \begin{enumerate}
      \item[a.] \( x^2 - 2xe^{-x} + e^{-2x} = 0 \), [0, 1]

        \underline{Sol}:\\
        Newton's method sequence \( \{p_n\} \) with \( p_0 = 0.5 \):
        \[
          \begin{array}{l}
            p_0 = 0.5 \\
            p_1 \approx 0.533338 \\
            p_2 \approx 0.545753 \\
            p_3 \approx 0.551693 \\
          \end{array}
        \]
        Aitken's \( \Delta^2 \) method: \( q_n = p_n - \frac{(p_{n+1}
        - p_n)^2}{(p_{n+2} - 2p_{n+1} + p_n)} \)
        \[
          \begin{array}{l}
            q_0 = p_0 - \frac{(p_1 - p_0)^2}{(p_2 - 2p_1 + p_0)}
            \approx 0.557521 \\
            q_1 = p_1 - \frac{(p_2 - p_1)^2}{(p_3 - 2p_2 + p_1)}
            \approx 0.557528 \\
          \end{array}
        \]
        \( |q_1 - q_0| \approx 0.000007 < 10^{-4} \). Stop at \( q_1 \).
        Root for part a: \( \boxed{0.55753} \)

      \item[c.] \( x^3 - 3x^2 (2^{-x}) + 3x (4^{-x}) - 8^{-x} = 0 \), [0, 1]

        Newton's method sequence \( \{p_n\} \) with \( p_0 = 0.5 \):
        \[
          \begin{array}{l}
            p_0 = 0.5 \\
            p_1 \approx 0.453476 \\
            p_2 \approx 0.447235 \\
            p_3 \approx 0.446729 \\
          \end{array}
        \]
        Aitken's \( \Delta^2 \) method: \( q_n = p_n - \frac{(p_{n+1}
        - p_n)^2}{(p_{n+2} - 2p_{n+1} + p_n)} \)
        \[
          \begin{array}{l}
            q_0 = p_0 - \frac{(p_1 - p_0)^2}{(p_2 - 2p_1 + p_0)}
            \approx 0.446734 \\
            q_1 = p_1 - \frac{(p_2 - p_1)^2}{(p_3 - 2p_2 + p_1)}
            \approx 0.446715 \\
            q_2 = p_2 - \frac{(p_3 - p_2)^2}{(p_4 - 2p_3 + p_2)},
            \text{ need } p_4 \approx 0.446715 \\
          \end{array}
        \]
        \( |q_1 - q_0| \approx 0.000019 > 10^{-4} \). Need more iterations.
        Since \( q_1 \) and \( q_2 \) are very close to \( q_1
        \approx 0.446715 \), we approximate root as \( q_1 \).

        Root for part c: \( \boxed{0.44672} \)
    \end{enumerate}

  \item[3.] Consider the function \( f(x) = e^{6x} + 3 (\ln 2)^2
    e^{2x} - (\ln 8)e^{4x} - (\ln 2)^3 \). Use Newton's method with
    \( p_0 = 0 \) to approximate a zero of \( f \). Generate terms
    until \( |p_{n+1} - p_n| < 0.0002 \). Construct Aitken's \(
    \Delta^2 \) sequence \( \lbrace q_n \rbrace \). Is the convergence improved?

    \underline{Sol}:\\
    Let \( f(x) = e^{6x} + 3 (\ln 2)^2 e^{2x} - (\ln 8)e^{4x} - (\ln
    2)^3 \) and \( f'(x) = 6e^{6x} + 6 (\ln 2)^2 e^{2x} - 4 (\ln 8)e^{4x} \).
    Newton's method iteration: \( p_{n+1} = p_n -
    \frac{f(p_n)}{f'(p_n)} \). Start with \( p_0 = 0 \).
    Let \( L2 = \ln 2 \) and \( L8 = \ln 8 \). Then \( f(x) = e^{6x}
    + 3 L2^2 e^{2x} - L8 e^{4x} - L2^3 \) and \( f'(x) = 6e^{6x} + 6
    L2^2 e^{2x} - 4 L8 e^{4x} \).
    \[
      \begin{array}{l}
        p_0 = 0 \\
        f(p_0) = 1 + 3 (\ln 2)^2 - \ln 8 - (\ln 2)^3 \\
        f'(p_0) = 6 + 6 (\ln 2)^2 - 4 \ln 8 \\
        p_1 = p_0 - \frac{f(p_0)}{f'(p_0)} = - \frac{1 + 3 (\ln 2)^2
        - \ln 8 - (\ln 2)^3}{6 + 6 (\ln 2)^2 - 4 \ln 8} \approx
        -2.06265 \times 10^{-7} \\
        |p_1 - p_0| = |p_1| \approx 2.06265 \times 10^{-7} < 0.0002 \\
      \end{array}
    \]
    Since \( |p_1 - p_0| < 0.0002 \), we stop at \( p_1 \).  \( p_1
    \approx -2.06265 \times 10^{-7} \).

    Construct Aitken's \( \Delta^2 \) sequence \( \{ q_n \} \). We
    need \( p_2 \) for \( q_0 \).
    \[
      \begin{array}{l}
        p_2 = p_1 - \frac{f(p_1)}{f'(p_1)} \\
      \end{array}
    \]
    Since \( p_1 \) is very close to 0 and \( f(0) \approx 0 \), \(
    p_2 \) will be very close to \( p_1 \).  For practical purposes,
    \( p_1 \approx p_2 \approx ... \approx 0 \).

    Aitken's \( \Delta^2 \) method: \( q_n = p_n - \frac{(p_{n+1} -
    p_n)^2}{(p_{n+2} - 2p_{n+1} + p_n)} \)
    \[
      q_0 = p_0 - \frac{(p_1 - p_0)^2}{(p_2 - 2p_1 + p_0)} = 0 -
      \frac{(p_1 - 0)^2}{(p_2 - 2p_1 + 0)} = - \frac{p_1^2}{p_2 - 2p_1}
    \]
    Since \( p_1 \approx p_2 \approx -2.06265 \times 10^{-7} \),
    let's use \( p_2 \approx p_1 \).
    \[
      q_0 \approx - \frac{p_1^2}{p_1 - 2p_1} = - \frac{p_1^2}{-p_1} =
      p_1 \approx -2.06265 \times 10^{-7}
    \]
    In this case, Aitken's method does not significantly improve the
    first approximation, as Newton's method already converges very
    rapidly from \( p_0 = 0 \). The convergence is already very fast,
    so acceleration by Aitken's method is not visibly significant in
    the first term \( q_0 \).

    Approximation of zero using Newton's method: \( \boxed{-2.06265
    \times 10^{-7}} \) Convergence is already very fast; Aitken's \(
    \Delta^2 \) method does not show significant improvement in the first term.

  \item[5.] (i) Show that the following sequences \( \lbrace p_n
    \rbrace \) converge linearly to \( p = 0 \). (ii) How large must
    \(n\) be before \( |p_n - p| \le 5 \times 10^{-2} \)? (iii) Use
    Aitken's \(\Delta^2\) method to generate a sequence \(
    lbrace q_n \rbrace\) until \( |q_n - p| \le 5 \times 10^{-2} \).
    \begin{enumerate}
      \item[a.] \( p_n = \frac{1}{n} \), for \( n \ge 1 \)

        \underline{Sol}:\\

        (i) Linear convergence:
        \[
          \lim_{n \to \infty} \frac{|p_{n+1} - 0|}{|p_n - 0|} =
          \lim_{n \to \infty} \frac{1/(n+1)}{1/n} = \lim_{n \to
          \infty} \frac{n}{n+1} = 1
        \]
        Linear convergence to \( p=0 \).

        (ii) Find \( n \) for \( |p_n - 0| \le 5 \times 10^{-2} \):
        \[
          \frac{1}{n} \le 0.05 = \frac{1}{20} \implies n \ge 20
        \]
        \( n = 20 \) needed.

        (iii) Aitken's \( \Delta^2 \) method: \( q_n = \frac{1}{2(n+1)} \)
        \[
          \begin{array}{l}
            q_1 = \frac{1}{2(1+1)} = \frac{1}{4} = 0.25 \\
            q_2 = \frac{1}{2(2+1)} = \frac{1}{6} \approx 0.16667 \\
            q_3 = \frac{1}{2(3+1)} = \frac{1}{8} = 0.125 \\
            q_4 = \frac{1}{2(4+1)} = \frac{1}{10} = 0.1 \\
            q_5 = \frac{1}{2(5+1)} = \frac{1}{12} \approx 0.08333 \\
            q_6 = \frac{1}{2(6+1)} = \frac{1}{14} \approx 0.07143 \\
            q_7 = \frac{1}{2(7+1)} = \frac{1}{16} = 0.0625 \\
            q_8 = \frac{1}{2(8+1)} = \frac{1}{18} \approx 0.05556 \\
            q_9 = \frac{1}{2(9+1)} = \frac{1}{20} = 0.05 \\
            q_{10} = \frac{1}{2(10+1)} = \frac{1}{22} \approx 0.04545 < 0.05 \\
          \end{array}
        \]
        Need \( q_{10} \) for \( |q_n| \le 5 \times 10^{-2} \).


      \item[b.] \( p_n = \frac{1}{n^2} \), for \( n \ge 1 \)

        (i) Linear convergence:
        \[
          \lim_{n \to \infty} \frac{|p_{n+1} - 0|}{|p_n - 0|} =
          \lim_{n \to \infty} \frac{1/(n+1)^2}{1/n^2} = \lim_{n \to
          \infty} \left(\frac{n}{n+1}\right)^2 = 1
        \]
        Linear convergence to \( p=0 \).

        (ii) Find \( n \) for \( |p_n - 0| \le 5 \times 10^{-2} \):
        \[
          \frac{1}{n^2} \le 0.05 = \frac{1}{20} \implies n^2 \ge 20
          \implies n \ge \sqrt{20} \approx 4.47
        \]
        \( n = 5 \) needed.

        (iii) Aitken's \( \Delta^2 \) method: \( q_1 = p_1 -
        \frac{(p_2 - p_1)^2}{(p_3 - 2p_2 + p_1)} \)
        \[
          \begin{array}{l}
            p_1 = 1, p_2 = 0.25, p_3 \approx 0.1111 \\
            q_1 \approx 0.0795 \\
            p_2 = 0.25, p_3 \approx 0.1111, p_4 = 1/16 = 0.0625 \\
            q_2 = 0.25 - \frac{(0.1111 - 0.25)^2}{(0.0625 - 2 \times
            0.1111 + 0.25)} \approx 0.03635 \\
          \end{array}
        \]
        \( |q_2| \approx 0.03635 < 0.05 \). Need \( q_2 \) for \(
        |q_n| \le 5 \times 10^{-2} \).

        \textbf{Answers:}

        Part a: (i) Linear, (ii) \( n = 20 \), (iii) \( q_{10}
        \approx 0.04545 \)

        Part b: (i) Linear, (ii) \( n = 5 \), (iii) \( q_2 \approx 0.03635 \)

    \end{enumerate}
\end{enumerate}
